\documentclass[]{book}
\usepackage{lmodern}
\usepackage{amssymb,amsmath}
\usepackage{ifxetex,ifluatex}
\usepackage{fixltx2e} % provides \textsubscript
\ifnum 0\ifxetex 1\fi\ifluatex 1\fi=0 % if pdftex
  \usepackage[T1]{fontenc}
  \usepackage[utf8]{inputenc}
\else % if luatex or xelatex
  \ifxetex
    \usepackage{mathspec}
  \else
    \usepackage{fontspec}
  \fi
  \defaultfontfeatures{Ligatures=TeX,Scale=MatchLowercase}
\fi
% use upquote if available, for straight quotes in verbatim environments
\IfFileExists{upquote.sty}{\usepackage{upquote}}{}
% use microtype if available
\IfFileExists{microtype.sty}{%
\usepackage{microtype}
\UseMicrotypeSet[protrusion]{basicmath} % disable protrusion for tt fonts
}{}
\usepackage[margin=1in]{geometry}
\usepackage{hyperref}
\hypersetup{unicode=true,
            pdftitle={R Lab for Statistical Computing},
            pdfborder={0 0 0},
            breaklinks=true}
\urlstyle{same}  % don't use monospace font for urls
\usepackage{natbib}
\bibliographystyle{apalike}
\usepackage{color}
\usepackage{fancyvrb}
\newcommand{\VerbBar}{|}
\newcommand{\VERB}{\Verb[commandchars=\\\{\}]}
\DefineVerbatimEnvironment{Highlighting}{Verbatim}{commandchars=\\\{\}}
% Add ',fontsize=\small' for more characters per line
\usepackage{framed}
\definecolor{shadecolor}{RGB}{248,248,248}
\newenvironment{Shaded}{\begin{snugshade}}{\end{snugshade}}
\newcommand{\AlertTok}[1]{\textcolor[rgb]{0.94,0.16,0.16}{#1}}
\newcommand{\AnnotationTok}[1]{\textcolor[rgb]{0.56,0.35,0.01}{\textbf{\textit{#1}}}}
\newcommand{\AttributeTok}[1]{\textcolor[rgb]{0.77,0.63,0.00}{#1}}
\newcommand{\BaseNTok}[1]{\textcolor[rgb]{0.00,0.00,0.81}{#1}}
\newcommand{\BuiltInTok}[1]{#1}
\newcommand{\CharTok}[1]{\textcolor[rgb]{0.31,0.60,0.02}{#1}}
\newcommand{\CommentTok}[1]{\textcolor[rgb]{0.56,0.35,0.01}{\textit{#1}}}
\newcommand{\CommentVarTok}[1]{\textcolor[rgb]{0.56,0.35,0.01}{\textbf{\textit{#1}}}}
\newcommand{\ConstantTok}[1]{\textcolor[rgb]{0.00,0.00,0.00}{#1}}
\newcommand{\ControlFlowTok}[1]{\textcolor[rgb]{0.13,0.29,0.53}{\textbf{#1}}}
\newcommand{\DataTypeTok}[1]{\textcolor[rgb]{0.13,0.29,0.53}{#1}}
\newcommand{\DecValTok}[1]{\textcolor[rgb]{0.00,0.00,0.81}{#1}}
\newcommand{\DocumentationTok}[1]{\textcolor[rgb]{0.56,0.35,0.01}{\textbf{\textit{#1}}}}
\newcommand{\ErrorTok}[1]{\textcolor[rgb]{0.64,0.00,0.00}{\textbf{#1}}}
\newcommand{\ExtensionTok}[1]{#1}
\newcommand{\FloatTok}[1]{\textcolor[rgb]{0.00,0.00,0.81}{#1}}
\newcommand{\FunctionTok}[1]{\textcolor[rgb]{0.00,0.00,0.00}{#1}}
\newcommand{\ImportTok}[1]{#1}
\newcommand{\InformationTok}[1]{\textcolor[rgb]{0.56,0.35,0.01}{\textbf{\textit{#1}}}}
\newcommand{\KeywordTok}[1]{\textcolor[rgb]{0.13,0.29,0.53}{\textbf{#1}}}
\newcommand{\NormalTok}[1]{#1}
\newcommand{\OperatorTok}[1]{\textcolor[rgb]{0.81,0.36,0.00}{\textbf{#1}}}
\newcommand{\OtherTok}[1]{\textcolor[rgb]{0.56,0.35,0.01}{#1}}
\newcommand{\PreprocessorTok}[1]{\textcolor[rgb]{0.56,0.35,0.01}{\textit{#1}}}
\newcommand{\RegionMarkerTok}[1]{#1}
\newcommand{\SpecialCharTok}[1]{\textcolor[rgb]{0.00,0.00,0.00}{#1}}
\newcommand{\SpecialStringTok}[1]{\textcolor[rgb]{0.31,0.60,0.02}{#1}}
\newcommand{\StringTok}[1]{\textcolor[rgb]{0.31,0.60,0.02}{#1}}
\newcommand{\VariableTok}[1]{\textcolor[rgb]{0.00,0.00,0.00}{#1}}
\newcommand{\VerbatimStringTok}[1]{\textcolor[rgb]{0.31,0.60,0.02}{#1}}
\newcommand{\WarningTok}[1]{\textcolor[rgb]{0.56,0.35,0.01}{\textbf{\textit{#1}}}}
\usepackage{longtable,booktabs}
\usepackage{graphicx,grffile}
\makeatletter
\def\maxwidth{\ifdim\Gin@nat@width>\linewidth\linewidth\else\Gin@nat@width\fi}
\def\maxheight{\ifdim\Gin@nat@height>\textheight\textheight\else\Gin@nat@height\fi}
\makeatother
% Scale images if necessary, so that they will not overflow the page
% margins by default, and it is still possible to overwrite the defaults
% using explicit options in \includegraphics[width, height, ...]{}
\setkeys{Gin}{width=\maxwidth,height=\maxheight,keepaspectratio}
\IfFileExists{parskip.sty}{%
\usepackage{parskip}
}{% else
\setlength{\parindent}{0pt}
\setlength{\parskip}{6pt plus 2pt minus 1pt}
}
\setlength{\emergencystretch}{3em}  % prevent overfull lines
\providecommand{\tightlist}{%
  \setlength{\itemsep}{0pt}\setlength{\parskip}{0pt}}
\setcounter{secnumdepth}{5}
% Redefines (sub)paragraphs to behave more like sections
\ifx\paragraph\undefined\else
\let\oldparagraph\paragraph
\renewcommand{\paragraph}[1]{\oldparagraph{#1}\mbox{}}
\fi
\ifx\subparagraph\undefined\else
\let\oldsubparagraph\subparagraph
\renewcommand{\subparagraph}[1]{\oldsubparagraph{#1}\mbox{}}
\fi

%%% Use protect on footnotes to avoid problems with footnotes in titles
\let\rmarkdownfootnote\footnote%
\def\footnote{\protect\rmarkdownfootnote}

%%% Change title format to be more compact
\usepackage{titling}

% Create subtitle command for use in maketitle
\providecommand{\subtitle}[1]{
  \posttitle{
    \begin{center}\large#1\end{center}
    }
}

\setlength{\droptitle}{-2em}

  \title{R Lab for Statistical Computing}
    \pretitle{\vspace{\droptitle}\centering\huge}
  \posttitle{\par}
    \author{\href{https://github.com/ygeunkim}{Young-geun Kim}\\
\href{https://stat.skku.edu/stat/index.jsp}{Department of Statistics}, \href{https://www.skku.edu/skku/index.do}{SKKU}\\
\href{mailto:dudrms33@g.skku.edu}{\nolinkurl{dudrms33@g.skku.edu}}}
    \preauthor{\centering\large\emph}
  \postauthor{\par}
      \predate{\centering\large\emph}
  \postdate{\par}
    \date{16 Apr, 2019}

\usepackage{booktabs}
\usepackage{longtable}
\usepackage{float}
\usepackage{pdfpages}
\usepackage{hyperref}
\usepackage[boxruled, linesnumbered]{algorithm2e}
\IncMargin{1.5em}

\newcommand{\iid}{\stackrel{iid}{\sim}}
\newcommand{\indep}{\stackrel{indep}{\sim}}
\newcommand{\hsim}{\stackrel{H_0}{\sim}}
\newcommand{\ind}{\perp\!\!\!\perp}
\newcommand{\R}{\mathbb{R}}
\DeclareMathOperator*{\argmin}{argmin}

\let\oldmaketitle\maketitle
\AtBeginDocument{\let\maketitle\relax}

\usepackage{amsthm}
\newtheorem{theorem}{Theorem}[chapter]
\newtheorem{lemma}{Lemma}[chapter]
\newtheorem{corollary}{Corollary}[chapter]
\newtheorem{proposition}{Proposition}[chapter]
\newtheorem{conjecture}{Conjecture}[chapter]
\theoremstyle{definition}
\newtheorem{definition}{Definition}[chapter]
\theoremstyle{definition}
\newtheorem{example}{Example}[chapter]
\theoremstyle{definition}
\newtheorem{exercise}{Exercise}[chapter]
\theoremstyle{remark}
\newtheorem*{remark}{Remark}
\newtheorem*{solution}{Solution}
\let\BeginKnitrBlock\begin \let\EndKnitrBlock\end
\begin{document}
\maketitle

\begin{titlepage}
  \includepdf{cover.pdf}
\end{titlepage}

\let\maketitle\oldmaketitle
\maketitle

{
\setcounter{tocdepth}{1}
\tableofcontents
}
\hypertarget{welcome}{%
\chapter*{Welcome}\label{welcome}}
\addcontentsline{toc}{chapter}{Welcome}

Statistical computing mainly treats useful simulation methods.

\begin{Shaded}
\begin{Highlighting}[]
\KeywordTok{library}\NormalTok{(tidyverse)}
\end{Highlighting}
\end{Shaded}

\texttt{tidyverse} package family will be used in every chapter. Loading step is in \texttt{\_common.R}, so it is not included in the text. Sometimes \texttt{data.table} library will be called for efficiency.

\hypertarget{statistical-computing}{%
\section*{Statistical Computing}\label{statistical-computing}}
\addcontentsline{toc}{section}{Statistical Computing}

We first look at \emph{random generation} methods. Lots of simulation methods are built based on this random numbers.

\hypertarget{sampling-from-a-fininte-population}{%
\subsection*{Sampling from a fininte population}\label{sampling-from-a-fininte-population}}
\addcontentsline{toc}{subsection}{Sampling from a fininte population}

Generating random numbers is like sampling. From finite population, we can sample data with or without replacement. For example of sampling with replacement, we toss coins 10 times.

\begin{Shaded}
\begin{Highlighting}[]
\KeywordTok{sample}\NormalTok{(}\DecValTok{0}\OperatorTok{:}\DecValTok{1}\NormalTok{, }\DataTypeTok{size =} \DecValTok{10}\NormalTok{, }\DataTypeTok{replace =} \OtherTok{TRUE}\NormalTok{)}
\CommentTok{#>  [1] 1 0 0 1 0 1 1 0 1 1}
\end{Highlighting}
\end{Shaded}

Sampling without replacement: Choose some lottery numbers which consist of 1 to 100.

\begin{Shaded}
\begin{Highlighting}[]
\KeywordTok{sample}\NormalTok{(}\DecValTok{1}\OperatorTok{:}\DecValTok{100}\NormalTok{, }\DataTypeTok{size =} \DecValTok{6}\NormalTok{, }\DataTypeTok{replace =} \OtherTok{FALSE}\NormalTok{)}
\CommentTok{#> [1] 61 83 50 74 34 35}
\end{Highlighting}
\end{Shaded}

\hypertarget{random-generators-of-common-probability-distributions}{%
\subsection*{Random generators of common probability distributions}\label{random-generators-of-common-probability-distributions}}
\addcontentsline{toc}{subsection}{Random generators of common probability distributions}

\texttt{R} provides some functions which generate random numbers following famous distributions. Although we will learn some skills generating these numbers in basis levels, these functions do the same thing more elegantly.

\begin{Shaded}
\begin{Highlighting}[]
\KeywordTok{gg_curve}\NormalTok{(dbeta, }\DataTypeTok{from =} \DecValTok{0}\NormalTok{, }\DataTypeTok{to =} \DecValTok{1}\NormalTok{, }\DataTypeTok{args =} \KeywordTok{list}\NormalTok{(}\DataTypeTok{shape1 =} \DecValTok{3}\NormalTok{, }\DataTypeTok{shape2 =} \DecValTok{2}\NormalTok{)) }\OperatorTok{+}
\StringTok{  }\KeywordTok{geom_histogram}\NormalTok{(}
    \DataTypeTok{data =} \KeywordTok{tibble}\NormalTok{(}
      \DataTypeTok{rand =} \KeywordTok{rbeta}\NormalTok{(}\DecValTok{1000}\NormalTok{, }\DecValTok{3}\NormalTok{, }\DecValTok{2}\NormalTok{),}
      \DataTypeTok{idx =} \KeywordTok{seq}\NormalTok{(}\DecValTok{0}\NormalTok{, }\DecValTok{1}\NormalTok{, }\DataTypeTok{length.out =} \DecValTok{1000}\NormalTok{)}
\NormalTok{    ),}
    \KeywordTok{aes}\NormalTok{(}\DataTypeTok{x =}\NormalTok{ rand, }\DataTypeTok{y =}\NormalTok{ ..density..),}
    \DataTypeTok{position =} \StringTok{"identity"}\NormalTok{,}
    \DataTypeTok{bins =} \DecValTok{30}\NormalTok{,}
    \DataTypeTok{alpha =} \FloatTok{.45}\NormalTok{,}
    \DataTypeTok{fill =} \KeywordTok{gg_hcl}\NormalTok{(}\DecValTok{1}\NormalTok{)}
\NormalTok{  )}
\end{Highlighting}
\end{Shaded}

\begin{figure}[H]

{\centering \includegraphics[width=0.7\linewidth]{statistical-computing_files/figure-latex/dbrb-1} 

}

\caption{Beta(3,2) random numbers}\label{fig:dbrb}
\end{figure}

Figure \ref{fig:dbrb} shows that \texttt{rbeta()} function generate random numbers very well. Histogram is of the random number, and the curve is the true beta distribution.

\hypertarget{rvar}{%
\chapter{Methods for Generating Random Variables}\label{rvar}}

\hypertarget{introduction}{%
\section{Introduction}\label{introduction}}

Most of the methods so-called \emph{computational statistics} requires generation of random variables from specified probability distribution. In hand, we can spin wheels, roll a dice, or shuffle cards. The results are chosen randomly. However, we want the same things with computer. Here, \texttt{r}. As we know, computer cannot generate complete uniform random numbers. Instead, we generate \textbf{pseudo-random} numbers.

\hypertarget{pseudo-random-numbers}{%
\section{Pseudo-random Numbers}\label{pseudo-random-numbers}}

\BeginKnitrBlock{definition}[Pseudo-random numbers]
\protect\hypertarget{def:unnamed-chunk-6}{}{\label{def:unnamed-chunk-6} \iffalse (Pseudo-random numbers) \fi{} }Sequence of values generated deterministically which have all the appearances of being independent \(unif(0, 1)\) random variables, i.e.

\[x_1, x_2, \ldots, x_n \stackrel{iid}{\sim} unif(0, 1)\]
\EndKnitrBlock{definition}

\begin{itemize}
\tightlist
\item
  behave \emph{as if} following \(unif(0, 1)\)
\item
  typically generated from an \emph{initial seed}
\end{itemize}

\hypertarget{linear-congruential-generator}{%
\subsection{Linear congruential generator}\label{linear-congruential-generator}}

Then \(u_1, u_2, \ldots, u_n \sim unif(0, 1)\)

\begin{algorithm}[H] \label{alg:alglcg}
  \SetAlgoLined
  \SetKwInOut{Input}{input}
  \SetKwInOut{Output}{output}
  \Input{$a, c \in \mathbb{Z}_{+}$ and modulus $m$}
  Initialize $x_0$\;
  \For{$i \leftarrow 1$ \KwTo $n$}{
    $x_i = (a x_{i - 1} + c) \mod m$\;
  }
  $u_i = \frac{x_i}{m} \in (0, 1)$\;
  \Output{$u_1, u_2, \ldots, u_n \sim unif(0, 1)$}
  \caption{Linear congruential generator}
\end{algorithm}

\begin{Shaded}
\begin{Highlighting}[]
\NormalTok{lcg <-}\StringTok{ }\ControlFlowTok{function}\NormalTok{(n, seed, a, b, m) \{}
\NormalTok{  x <-}\StringTok{ }\KeywordTok{rep}\NormalTok{(seed, n }\OperatorTok{+}\StringTok{ }\DecValTok{1}\NormalTok{)}
  \ControlFlowTok{for}\NormalTok{ (i }\ControlFlowTok{in} \DecValTok{1}\OperatorTok{:}\NormalTok{n) \{}
\NormalTok{    x[i }\OperatorTok{+}\StringTok{ }\DecValTok{1}\NormalTok{] <-}\StringTok{ }\NormalTok{(a }\OperatorTok{*}\StringTok{ }\NormalTok{x[i] }\OperatorTok{+}\StringTok{ }\NormalTok{b) }\OperatorTok\StringTok{ }\NormalTok{m}
\NormalTok{  \}}
\NormalTok{  x[}\OperatorTok{-}\DecValTok{1}\NormalTok{] }\OperatorTok{/}\StringTok{ }\NormalTok{m}
\NormalTok{\}}
\end{Highlighting}
\end{Shaded}

\begin{Shaded}
\begin{Highlighting}[]
\KeywordTok{tibble}\NormalTok{(}
  \DataTypeTok{x =} \KeywordTok{lcg}\NormalTok{(}\DecValTok{1000}\NormalTok{, }\DecValTok{0}\NormalTok{, }\DecValTok{1664525}\NormalTok{, }\DecValTok{1013904223}\NormalTok{, }\DecValTok{2}\OperatorTok{^}\DecValTok{32}\NormalTok{)}
\NormalTok{) }\OperatorTok\StringTok{ }
\StringTok{  }\KeywordTok{ggplot}\NormalTok{(}\KeywordTok{aes}\NormalTok{(}\DataTypeTok{x =}\NormalTok{ x)) }\OperatorTok{+}
\StringTok{  }\KeywordTok{geom_histogram}\NormalTok{(}\KeywordTok{aes}\NormalTok{(}\DataTypeTok{y =}\NormalTok{ ..density..), }\DataTypeTok{bins =} \DecValTok{30}\NormalTok{, }\DataTypeTok{col =} \KeywordTok{gg_hcl}\NormalTok{(}\DecValTok{1}\NormalTok{))}
\end{Highlighting}
\end{Shaded}

\begin{center}\includegraphics[width=0.7\linewidth]{statistical-computing_files/figure-latex/unnamed-chunk-8-1} \end{center}

\hypertarget{multiplicative-congruential-generator}{%
\subsection{Multiplicative congruential generator}\label{multiplicative-congruential-generator}}

As we can expect from its name, this is congruential generator with \(c = 0\).

\begin{algorithm}[H] \label{alg:algmcg}
  \SetAlgoLined
  \SetKwInOut{Input}{input}
  \SetKwInOut{Output}{output}
  \Input{$a, \in \mathbb{Z}_{+}$ and modulus $m$}
  Initialize $x_0$\;
  \For{$i \leftarrow 1$ \KwTo $n$}{
    $x_i = a x_{i - 1} \mod m$\;
  }
  $u_i = \frac{x_i}{m} \in (0, 1)$\;
  \Output{$u_1, u_2, \ldots, u_n \sim unif(0, 1)$}
  \caption{Multiplicative congruential generator}
\end{algorithm}

We just set \texttt{b\ =\ 0} in our \texttt{lcg()} function. The \textbf{seed must not be zero}.

\begin{Shaded}
\begin{Highlighting}[]
\KeywordTok{tibble}\NormalTok{(}
  \DataTypeTok{x =} \KeywordTok{lcg}\NormalTok{(}\DecValTok{1000}\NormalTok{, }\DecValTok{5}\NormalTok{, }\DecValTok{1664525}\NormalTok{, }\DecValTok{0}\NormalTok{, }\DecValTok{2}\OperatorTok{^}\DecValTok{32}\NormalTok{)}
\NormalTok{) }\OperatorTok\StringTok{ }
\StringTok{  }\KeywordTok{ggplot}\NormalTok{(}\KeywordTok{aes}\NormalTok{(}\DataTypeTok{x =}\NormalTok{ x)) }\OperatorTok{+}
\StringTok{  }\KeywordTok{geom_histogram}\NormalTok{(}\KeywordTok{aes}\NormalTok{(}\DataTypeTok{y =}\NormalTok{ ..density..), }\DataTypeTok{bins =} \DecValTok{30}\NormalTok{, }\DataTypeTok{col =} \KeywordTok{gg_hcl}\NormalTok{(}\DecValTok{1}\NormalTok{))}
\end{Highlighting}
\end{Shaded}

\begin{center}\includegraphics[width=0.7\linewidth]{statistical-computing_files/figure-latex/unnamed-chunk-9-1} \end{center}

\hypertarget{cycle}{%
\subsection{Cycle}\label{cycle}}

Generate LCG \(n = 32\) with \(a = 1\), \(c = 1\), and \(m = 16\) from the seed \(x_0 = 0\).

\begin{Shaded}
\begin{Highlighting}[]
\KeywordTok{lcg}\NormalTok{(}\DecValTok{32}\NormalTok{, }\DecValTok{0}\NormalTok{, }\DecValTok{1}\NormalTok{, }\DecValTok{1}\NormalTok{, }\DecValTok{16}\NormalTok{)}
\CommentTok{#>  [1] 0.0625 0.1250 0.1875 0.2500 0.3125 0.3750 0.4375 0.5000 0.5625 0.6250}
\CommentTok{#> [11] 0.6875 0.7500 0.8125 0.8750 0.9375 0.0000 0.0625 0.1250 0.1875 0.2500}
\CommentTok{#> [21] 0.3125 0.3750 0.4375 0.5000 0.5625 0.6250 0.6875 0.7500 0.8125 0.8750}
\CommentTok{#> [31] 0.9375 0.0000}
\end{Highlighting}
\end{Shaded}

Observe that we have the cycle after \(m\)-th number. Against this problem, we give different seed from every \((im + 1)\)th random number.

\hypertarget{the-inverse-transform-method}{%
\section{The Inverse Transform Method}\label{the-inverse-transform-method}}

\BeginKnitrBlock{definition}[Inverse of CDF]
\protect\hypertarget{def:icdf}{}{\label{def:icdf} \iffalse (Inverse of CDF) \fi{} }Since some cdf \(F_X\) is not strictly increasing, we difine \(F_X^{-1}(y)\) for \(0 < y < 1\) by

\[F_{X}^{-1}(y) := inf \{ x : F_X(x) \ge y \}\]
\EndKnitrBlock{definition}

Using this definition, we can get the following theorem.

\BeginKnitrBlock{theorem}[Probability Integral Transformation]
\protect\hypertarget{thm:probint}{}{\label{thm:probint} \iffalse (Probability Integral Transformation) \fi{} }If \(X\) is a continuous random variable with cdf \(F_(x)\), then
\[U \equiv F_X(X) \sim unif(0, 1)\]
\EndKnitrBlock{theorem}

\BeginKnitrBlock{proof}[Probability Integral Transformation]
\iffalse{} {Proof (Probability Integral Transformation). } \fi{}Let \(U \sim unif(0, 1)\). Then

\begin{equation*}
  \begin{split}
    P(F_X^{-1}(U) \le x) & = P(\inf\{t : F_X(t) = U \} \le x) \\
    & = P(U \le F_X(x)) \\
    & = F_U(F_X(x)) \\
    & = F_X(x)
  \end{split}
\end{equation*}
\EndKnitrBlock{proof}

Thus, to generate \(n\) random variables \(\sim F_X\), we can use \emph{uniform random numbers}.

\begin{algorithm}[H] \label{alg:alginv1}
  \SetAlgoLined
  \SetKwInOut{Input}{input}
  \SetKwInOut{Output}{output}
  \Input{analytical form of $F_X^{-1}$}
  \For{$i \leftarrow 1$ \KwTo $n$}{
    $u_i \stackrel{iid}{\sim}unif(0, 1)$\;
    $x_i = F_X^{-1}(u_i)$\;
  }
  \Output{$x_1, x_2, \ldots, x_n \stackrel{iid}{\sim}F_X$}
  \caption{Inverse transformation method}
\end{algorithm}

Note that in \texttt{R}, vectorized operation would be better, i.e.~generate \texttt{runif(n)} and plug it into given inverse cdf.

\hypertarget{continuous-case}{%
\subsection{Continuous case}\label{continuous-case}}

Denote that the \emph{probability integral transformation} holds for a continuous variable. When generating continuous random variable, applying above algorithm might work.

\BeginKnitrBlock{example}[Exponential distribution]
\protect\hypertarget{exm:expon}{}{\label{exm:expon} \iffalse (Exponential distribution) \fi{} }If \(X \sim Exp(\lambda)\), then \(F_X(x) = 1 - e^{-\lambda x}\). We can derive the inverse function of cdf
\[F_X^{-1}(u) = \frac{1}{\lambda}\ln(1 - u)\]
\EndKnitrBlock{example}

Note that

\[U \sim unif(0, 1) \Leftrightarrow 1 - U \sim unif(0, 1)\]

Then we just can use \(U\) instead of \(1 - U\).

\begin{Shaded}
\begin{Highlighting}[]
\NormalTok{inv_exp <-}\StringTok{ }\ControlFlowTok{function}\NormalTok{(n, lambda) \{}
  \OperatorTok{-}\KeywordTok{log}\NormalTok{(}\KeywordTok{runif}\NormalTok{(n)) }\OperatorTok{/}\StringTok{ }\NormalTok{lambda}
\NormalTok{\}}
\end{Highlighting}
\end{Shaded}

If we generate \(x_1, \ldots, x_{500} \sim Exp(\lambda = 1)\),

\begin{Shaded}
\begin{Highlighting}[]
\KeywordTok{gg_curve}\NormalTok{(dexp, }\DataTypeTok{from =} \DecValTok{0}\NormalTok{, }\DataTypeTok{to =} \DecValTok{10}\NormalTok{) }\OperatorTok{+}
\StringTok{  }\KeywordTok{geom_histogram}\NormalTok{(}
    \DataTypeTok{data =} \KeywordTok{tibble}\NormalTok{(}\DataTypeTok{x =} \KeywordTok{inv_exp}\NormalTok{(}\DecValTok{500}\NormalTok{, }\DataTypeTok{lambda =} \DecValTok{1}\NormalTok{)),}
    \KeywordTok{aes}\NormalTok{(}\DataTypeTok{x =}\NormalTok{ x, }\DataTypeTok{y =}\NormalTok{ ..density..),}
    \DataTypeTok{bins =} \DecValTok{30}\NormalTok{,}
    \DataTypeTok{fill =} \KeywordTok{gg_hcl}\NormalTok{(}\DecValTok{1}\NormalTok{),}
    \DataTypeTok{alpha =} \FloatTok{.5}
\NormalTok{  )}
\end{Highlighting}
\end{Shaded}

\begin{figure}[H]

{\centering \includegraphics[width=0.7\linewidth]{statistical-computing_files/figure-latex/cdfexp-1} 

}

\caption{Inverse Transformation: Exp(1)}\label{fig:cdfexp}
\end{figure}

\hypertarget{discrete-case}{%
\subsection{Discrete case}\label{discrete-case}}

\begin{algorithm}[H] \label{alg:alginv2}
  \SetAlgoLined
  \SetKwInOut{Input}{input}
  \SetKwInOut{Output}{output}
  \Input{analytical form of $F_X$}
  \For{$i \leftarrow 1$ \KwTo $n$}{
    $u_i \stackrel{iid}{\sim}unif(0, 1)$\;
    Take $x_i$ s.t. $F_X(x_{i - 1}) < U \le F_X(x_i)$\;
  }
  \Output{$x_1, x_2, \ldots, x_n \stackrel{iid}{\sim}F_X$}
  \caption{Inverse transformation method in discrete case}
\end{algorithm}

\begin{longtable}[]{@{}lrrrrr@{}}
\caption{\label{tab:exdis}Example of a Discrete Random Variable}\tabularnewline
\toprule
\endhead
x & 0.0 & 1.0 & 2.0 & 3.0 & 4.0\tabularnewline
p & 0.1 & 0.2 & 0.2 & 0.2 & 0.3\tabularnewline
\bottomrule
\end{longtable}

\BeginKnitrBlock{example}[Discrete Random Variable]
\protect\hypertarget{exm:dismass}{}{\label{exm:dismass} \iffalse (Discrete Random Variable) \fi{} }Consider a discrete random variable \(X\) with a mass function as in Table \ref{tab:exdis}.
\EndKnitrBlock{example}

i.e.

\begin{figure}[H]

{\centering \includegraphics[width=0.7\linewidth]{statistical-computing_files/figure-latex/massfun-1} 

}

\caption{Probability Mass Function}\label{fig:massfun}
\end{figure}

Then we have the cdf

\begin{figure}[H]

{\centering \includegraphics[width=0.7\linewidth]{statistical-computing_files/figure-latex/cdfun-1} 

}

\caption{CDF of the Discrete Random Variable: Illustration for discrete case}\label{fig:cdfun}
\end{figure}

Remembering the algorithm, we can implement \texttt{dplyr::case\_when()} here.

\begin{Shaded}
\begin{Highlighting}[]
\NormalTok{rcustom <-}\StringTok{ }\ControlFlowTok{function}\NormalTok{(n) \{}
  \KeywordTok{tibble}\NormalTok{(}\DataTypeTok{u =} \KeywordTok{runif}\NormalTok{(n)) }\OperatorTok\StringTok{ }
\StringTok{    }\KeywordTok{mutate}\NormalTok{(}
      \DataTypeTok{x =} \KeywordTok{case_when}\NormalTok{(}
\NormalTok{        u }\OperatorTok{>}\StringTok{ }\DecValTok{0} \OperatorTok{&}\StringTok{ }\NormalTok{u }\OperatorTok{<=}\StringTok{ }\FloatTok{.1} \OperatorTok{~}\StringTok{ }\DecValTok{0}\NormalTok{,}
\NormalTok{        u }\OperatorTok{>}\StringTok{ }\FloatTok{.1} \OperatorTok{&}\StringTok{ }\NormalTok{u }\OperatorTok{<=}\StringTok{ }\FloatTok{.3} \OperatorTok{~}\StringTok{ }\DecValTok{1}\NormalTok{,}
\NormalTok{        u }\OperatorTok{>}\StringTok{ }\FloatTok{.3} \OperatorTok{&}\StringTok{ }\NormalTok{u }\OperatorTok{<=}\StringTok{ }\FloatTok{.5} \OperatorTok{~}\StringTok{ }\DecValTok{2}\NormalTok{,}
\NormalTok{        u }\OperatorTok{>}\StringTok{ }\FloatTok{.5} \OperatorTok{&}\StringTok{ }\NormalTok{u }\OperatorTok{<=}\StringTok{ }\FloatTok{.7} \OperatorTok{~}\StringTok{ }\DecValTok{3}\NormalTok{,}
        \OtherTok{TRUE} \OperatorTok{~}\StringTok{ }\DecValTok{4}
\NormalTok{      )}
\NormalTok{    ) }\OperatorTok\StringTok{ }
\StringTok{    }\KeywordTok{select}\NormalTok{(x) }\OperatorTok\StringTok{ }
\StringTok{    }\KeywordTok{pull}\NormalTok{()}
\NormalTok{\}}
\end{Highlighting}
\end{Shaded}

\begin{Shaded}
\begin{Highlighting}[]
\KeywordTok{tibble}\NormalTok{(}
  \DataTypeTok{x =} \KeywordTok{rcustom}\NormalTok{(}\DecValTok{100}\NormalTok{)}
\NormalTok{) }\OperatorTok\StringTok{ }
\StringTok{  }\KeywordTok{ggplot}\NormalTok{(}\KeywordTok{aes}\NormalTok{(}\DataTypeTok{x =}\NormalTok{ x)) }\OperatorTok{+}
\StringTok{  }\KeywordTok{geom_histogram}\NormalTok{(}\KeywordTok{aes}\NormalTok{(}\DataTypeTok{y =}\NormalTok{ ..ndensity..), }\DataTypeTok{binwidth =} \FloatTok{.1}\NormalTok{)}
\end{Highlighting}
\end{Shaded}

\begin{figure}[H]

{\centering \includegraphics[width=0.7\linewidth]{statistical-computing_files/figure-latex/randmass-1} 

}

\caption{Generated discrete random numbers}\label{fig:randmass}
\end{figure}

See Figure \ref{fig:massfun} and \ref{fig:randmass}. Comparing the two, the result can be said okay.

\hypertarget{problems-with-inverse-transformation}{%
\subsection{Problems with inverse transformation}\label{problems-with-inverse-transformation}}

Examples \ref{exm:expon} and \ref{exm:dismass}. We could generate these random numbers because we aware of

\begin{enumerate}
\def\labelenumi{\arabic{enumi}.}
\tightlist
\item
  analytical \(F_X\)
\item
  \(F^{-1}\)
\end{enumerate}

In practice, however, not all distribution have analytical \(F\). Numerical computing might be possible, but it is not efficient. There are other approaches.

\hypertarget{the-acceptance-rejection-method}{%
\section{The Acceptance-Rejection Method}\label{the-acceptance-rejection-method}}

Acceptance-rejection method does not require analytical form of cdf. What we need is our \emph{target} density (or mass) function and \emph{proposal} density (or mass) function. Target function is what we want to generate. Propsal function is of any random variable that is \emph{easy to generate random numbers}. From this approach, we can generate any distribution while computation is not efficient.

\begin{longtable}[]{@{}cc@{}}
\toprule
pdf or pmf & target or proposal\tabularnewline
\midrule
\endhead
\(f\) & target\tabularnewline
\(g\) & proposal - easy to generate random numbers\tabularnewline
\bottomrule
\end{longtable}

First of all, \(g\) should satisfy that

\[spt f \subseteq spt g\]

Next, for some (pre-specified) \(c > 0\)

\[\forall x \in spt f : \frac{f(x)}{g(x)} \le c\]

\begin{algorithm}[H] \label{alg:algar}
  \SetAlgoLined
  \SetKwInOut{Input}{input}
  \SetKwInOut{Output}{output}
  \Input{target $f$, proposal $g$, and $c$}
  \For{$i \leftarrow 1$ \KwTo $n$}{
    $Y \sim g(y)$\; \label{alg:argoto}
    $U \sim unif(0, 1) \perp\!\!\!\perp Y$\;
    \eIf{$U \le \frac{f(Y)}{cg(Y)}$}{
      Accept $x_i = Y$\;
    }{
      go to Line \ref{alg:argoto}\;
    }
  }
  \Output{$x_1, x_2, \ldots, x_n \stackrel{iid}{\sim}f(x)$}
  \caption{Acceptance-rejection algorithm}
\end{algorithm}

\hypertarget{efficiency}{%
\subsection{Efficiency}\label{efficiency}}

\begin{figure}[H]

{\centering \includegraphics[width=0.7\linewidth]{statistical-computing_files/figure-latex/arprop-1} 

}

\caption{Property of AR method}\label{fig:arprop}
\end{figure}

See Figure \ref{fig:arprop}. This illustrates the motivation of A-R method. Lower one is \(f(x)\) and the upper one is \(cg(x)\) which covers \(f\). We can see that

\[0 < \frac{f(x)}{cg(x)} \le 1\]

The algorithm takes random number from \(Y \sim g\) in each recursive step \(i\), which is represented as a line in the figure. At this value, the algorithm accept \(Y\) as random number of \(f\) if

\[U \le \frac{f(Y)}{cg(Y)}\]

Suppose that we choose a point at random on a line drawn in the figure \ref{fig:arprop}. If we get the red line, we accept. Otherwise, we reject. In other words, the \emph{colored area is where we reject the given value}. The smaller the area is, the more efficient the algorithm will be.

\BeginKnitrBlock{proposition}[Properties of A-R Method]
\protect\hypertarget{prp:arnote}{}{\label{prp:arnote} \iffalse (Properties of A-R Method) \fi{} }See Figure \ref{fig:arprop}.

\begin{enumerate}
  \item $\frac{f(Y)}{cg(Y)} \perp\!\!\!\perp U$
  \item $0 < \frac{f(x)}{cg(x)} \le 1$
  \item Let $N$ be the number of iterations needed to get an acceptance. Then $$N \sim Geo(p) \quad \text{where}\: p \equiv P\bigg(U \le \frac{f(Y)}{cg(Y)}\bigg)$$ and so
$$
\begin{cases}
  P(N = n) = p(1 - p)^{n - 1}I_{\{1, 2, \ldots \}}(n) \\
  E(N) = \text{average number of iterations} = \frac{1}{p}
\end{cases}
$$
  \item $X \sim Y \mid U \le \frac{f(Y)}{cg(Y)}$, i.e. $$P\bigg(Y \le y \mid U \le \frac{f(Y)}{cg(Y)}\bigg) = F_X(y)$$
\end{enumerate}
\EndKnitrBlock{proposition}

\BeginKnitrBlock{remark}[Efficiency]
\iffalse{} {Remark (Efficiency). } \fi{}Efficiency of the A-R method depends on \(p = P\bigg(U \le \frac{f(Y)}{cg(Y)}\bigg)\). In fact,

\[E(N) = \frac{1}{p} = c\]

The algorithm becomes efficient for small \(c\).
\EndKnitrBlock{remark}

\BeginKnitrBlock{proof}
\iffalse{} {Proof. } \fi{}Note that

\[P\bigg( U \le \frac{f(y)}{cg(y)}, Y = y \bigg) = P\bigg(Y \le \frac{g(y)}{cg(y)} \mid Y = y \bigg)P(Y = y)\]

Since \(U \sim unif(0, 1)\), \(P\bigg(Y \le \frac{g(y)}{cg(y)} \mid Y = y \bigg) = \frac{f(y)}{cg(y)}\).

By construction, \(P(Y = y) = g(y)\).

It follows that

\begin{equation*}
  \begin{split}
    p = P\bigg( U \le  \frac{f(y)}{cg(y)} \bigg) & = \int_{-\infty}^{\infty} P\bigg( U \le \frac{f(y)}{cg(y)}, Y = y \bigg) dy \\
    & = \int_{-\infty}^{\infty} \frac{f(y)}{cg(y)} g(y) dy \\
    & = \frac{1}{c} \int_{-\infty}^{\infty}f(y)dy \\
    & = \frac{1}{c}
  \end{split}
\end{equation*}

Hence,

\[E(N) = \frac{1}{p} = c\]

We can say that the method is efficient when the acceptance rate \(p\) is large, i.e.~\(c\) small.
\EndKnitrBlock{proof}

\BeginKnitrBlock{corollary}[Efficiency of A-R Method]
\protect\hypertarget{cor:argood}{}{\label{cor:argood} \iffalse (Efficiency of A-R Method) \fi{} }A-R method is efficient when

\(g(\cdot)\) is close to \(f(\cdot)\) and

have small \(c\).
\EndKnitrBlock{corollary}

\BeginKnitrBlock{corollary}[Choosing c]
\protect\hypertarget{cor:arc}{}{\label{cor:arc} \iffalse (Choosing c) \fi{} }To enhance the algorithm, we might choose \(c\) which satisfy

\[c = \max \bigg\{ \frac{f(x)}{g(x)} : x \in spt f \bigg\}\]
\EndKnitrBlock{corollary}

\hypertarget{examples}{%
\subsection{Examples}\label{examples}}

\BeginKnitrBlock{example}[Beta(a,b)]
\protect\hypertarget{exm:arbeta}{}{\label{exm:arbeta} \iffalse (Beta(a,b)) \fi{} }Let \(X \sim Beta(a, b)\). Then the pdf of \(X\) is given by

\[f(x) = \frac{1}{B(a, b)}x^{a - 1}(1 - x)^{b - 1}I_{(0, 1)}(x)\]
\EndKnitrBlock{example}

\BeginKnitrBlock{solution}[Generating Beta(a,b) with A-R method]
\iffalse{} {Solution (Generating Beta(a,b) with A-R method). } \fi{}Consider proposal density \(g(x) = I_{(0, 1)}(x)\), i.e.~\(unif(0, 1)\).

To determine the optimal \(c\) s.t.

\[c = \max \bigg\{ \frac{f(x)}{g(x)} : x \in (0, 1) \bigg\}\]

find the maximum of

\[\frac{f(x)}{g(x)} = \frac{1}{B(a, b)}x^{a - 1}(1 - x)^{b - 1}\]

Solve

\begin{equation*}
  \begin{split}
    \frac{d}{dx}\bigg(\frac{f(x)}{g(x)}\bigg) & = \frac{1}{B(a, b)}\Big( (a-1)x^{a-2}(1 - x)^{b - 1} - (b - 1)x^{a - 1}(1 - x)^{b - 2} \Big) \\
    & = \frac{x^{a - 2}(1 - x)^{b - 2}}{B(a, b)} \Big( (a - 1)(1 - x) - (b - 1)x \Big) \\
    & = \frac{x^{a - 2}(1 - x)^{b - 2}}{B(a, b)} \big( a - 1 - (a + b - 2)x \big) \quad = 0
  \end{split}
\end{equation*}

It follows that

\[\frac{f(x)}{g(x)} \le \frac{f(\frac{a - 1}{a + b - 2})}{g(\frac{a - 1}{a + b - 2})} = c\]

if \(\frac{a - 1}{a + b - 2} \neq 0, 1\)
\EndKnitrBlock{solution}

\begin{Shaded}
\begin{Highlighting}[]
\NormalTok{ar_beta <-}\StringTok{ }\ControlFlowTok{function}\NormalTok{(n, a, b) \{}
\NormalTok{  opt_x <-}\StringTok{ }\NormalTok{(a }\OperatorTok{-}\StringTok{ }\DecValTok{1}\NormalTok{) }\OperatorTok{/}\StringTok{ }\NormalTok{(a }\OperatorTok{+}\StringTok{ }\NormalTok{b }\OperatorTok{-}\StringTok{ }\DecValTok{2}\NormalTok{)}
\NormalTok{  opt_c <-}\StringTok{ }\KeywordTok{dbeta}\NormalTok{(opt_x, }\DataTypeTok{shape1 =}\NormalTok{ a, }\DataTypeTok{shape2 =}\NormalTok{ b) }\OperatorTok{/}\StringTok{ }\KeywordTok{dunif}\NormalTok{(opt_x)}
\NormalTok{  X <-}\StringTok{ }\OtherTok{NULL}
\NormalTok{  N <-}\StringTok{ }\DecValTok{0}
  \ControlFlowTok{while}\NormalTok{ (N }\OperatorTok{<=}\StringTok{ }\NormalTok{n) \{}
\NormalTok{    Y <-}\StringTok{ }\KeywordTok{runif}\NormalTok{(n)}
\NormalTok{    U <-}\StringTok{ }\KeywordTok{runif}\NormalTok{(n)}
\NormalTok{    X <-}\StringTok{ }\KeywordTok{c}\NormalTok{(X, Y[U }\OperatorTok{<=}\StringTok{ }\KeywordTok{dbeta}\NormalTok{(Y, }\DataTypeTok{shape1 =}\NormalTok{ a, }\DataTypeTok{shape2 =}\NormalTok{ b) }\OperatorTok{/}\StringTok{ }\NormalTok{opt_c])}
\NormalTok{    N <-}\StringTok{ }\KeywordTok{length}\NormalTok{(X)}
    \ControlFlowTok{if}\NormalTok{ ( N }\OperatorTok{>}\StringTok{ }\NormalTok{n ) X <-}\StringTok{ }\NormalTok{X[}\DecValTok{1}\OperatorTok{:}\NormalTok{n]}
\NormalTok{  \}}
\NormalTok{  X}
\NormalTok{\}}
\end{Highlighting}
\end{Shaded}

Now we try to compare this A-R function to \texttt{R} \texttt{rbeta} function.

\begin{Shaded}
\begin{Highlighting}[]
\NormalTok{gen_beta <-}
\StringTok{  }\KeywordTok{tibble}\NormalTok{(}
    \DataTypeTok{ar_rand =} \KeywordTok{ar_beta}\NormalTok{(}\DecValTok{1000}\NormalTok{, }\DecValTok{3}\NormalTok{, }\DecValTok{2}\NormalTok{),}
    \DataTypeTok{sam =} \KeywordTok{rbeta}\NormalTok{(}\DecValTok{1000}\NormalTok{, }\DecValTok{3}\NormalTok{, }\DecValTok{2}\NormalTok{)}
\NormalTok{  ) }\OperatorTok\StringTok{ }
\StringTok{  }\KeywordTok{gather}\NormalTok{(}\DataTypeTok{key =} \StringTok{"den"}\NormalTok{, }\DataTypeTok{value =} \StringTok{"value"}\NormalTok{)}
\end{Highlighting}
\end{Shaded}

\begin{Shaded}
\begin{Highlighting}[]
\KeywordTok{gg_curve}\NormalTok{(dbeta, }\DataTypeTok{from =} \DecValTok{0}\NormalTok{, }\DataTypeTok{to =} \DecValTok{1}\NormalTok{, }\DataTypeTok{args =} \KeywordTok{list}\NormalTok{(}\DataTypeTok{shape1 =} \DecValTok{3}\NormalTok{, }\DataTypeTok{shape2 =} \DecValTok{2}\NormalTok{)) }\OperatorTok{+}
\StringTok{  }\KeywordTok{geom_histogram}\NormalTok{(}
    \DataTypeTok{data =}\NormalTok{ gen_beta,}
    \KeywordTok{aes}\NormalTok{(}\DataTypeTok{x =}\NormalTok{ value, }\DataTypeTok{y =}\NormalTok{ ..density.., }\DataTypeTok{fill =}\NormalTok{ den),}
    \DataTypeTok{position =} \StringTok{"identity"}\NormalTok{,}
    \DataTypeTok{bins =} \DecValTok{30}\NormalTok{,}
    \DataTypeTok{alpha =} \FloatTok{.45}
\NormalTok{  ) }\OperatorTok{+}
\StringTok{  }\KeywordTok{scale_fill_discrete}\NormalTok{(}
    \DataTypeTok{name =} \StringTok{"random number"}\NormalTok{,}
    \DataTypeTok{labels =} \KeywordTok{c}\NormalTok{(}\StringTok{"AR"}\NormalTok{, }\StringTok{"rbeta"}\NormalTok{)}
\NormalTok{  )}
\end{Highlighting}
\end{Shaded}

\begin{figure}[H]

{\centering \includegraphics[width=0.7\linewidth]{statistical-computing_files/figure-latex/betahis-1} 

}

\caption{Beta(3,2) Random numbers from each function}\label{fig:betahis}
\end{figure}

In the Figure \ref{fig:betahis}, the both histograms are very close to the true density curve. To see more statistically, we can draw a Q-Q plot.

\begin{Shaded}
\begin{Highlighting}[]
\NormalTok{gen_beta }\OperatorTok\StringTok{ }
\StringTok{  }\KeywordTok{ggplot}\NormalTok{(}\KeywordTok{aes}\NormalTok{(}\DataTypeTok{sample =}\NormalTok{ value)) }\OperatorTok{+}
\StringTok{  }\KeywordTok{stat_qq_line}\NormalTok{(}
    \DataTypeTok{distribution =}\NormalTok{ stats}\OperatorTok{::}\NormalTok{qbeta,}
    \DataTypeTok{dparams =} \KeywordTok{list}\NormalTok{(}\DataTypeTok{shape1 =} \DecValTok{3}\NormalTok{, }\DataTypeTok{shape2 =} \DecValTok{2}\NormalTok{),}
    \DataTypeTok{col =} \KeywordTok{I}\NormalTok{(}\StringTok{"grey70"}\NormalTok{),}
    \DataTypeTok{size =} \FloatTok{3.5}
\NormalTok{  ) }\OperatorTok{+}
\StringTok{  }\KeywordTok{stat_qq}\NormalTok{(}
    \KeywordTok{aes}\NormalTok{(}\DataTypeTok{colour =}\NormalTok{ den),}
    \DataTypeTok{distribution =}\NormalTok{ stats}\OperatorTok{::}\NormalTok{qbeta,}
    \DataTypeTok{dparams =} \KeywordTok{list}\NormalTok{(}\DataTypeTok{shape1 =} \DecValTok{3}\NormalTok{, }\DataTypeTok{shape2 =} \DecValTok{2}\NormalTok{)}
\NormalTok{  ) }\OperatorTok{+}
\StringTok{  }\KeywordTok{scale_colour_discrete}\NormalTok{(}
    \DataTypeTok{name =} \StringTok{"random number"}\NormalTok{,}
    \DataTypeTok{labels =} \KeywordTok{c}\NormalTok{(}\StringTok{"AR"}\NormalTok{, }\StringTok{"rbeta"}\NormalTok{)}
\NormalTok{  )}
\end{Highlighting}
\end{Shaded}

\begin{figure}[H]

{\centering \includegraphics[width=0.7\linewidth]{statistical-computing_files/figure-latex/betaqq-1} 

}

\caption{Q-Q plot for Beta(3,2) random numbers}\label{fig:betaqq}
\end{figure}

See Figure \ref{fig:betaqq}. We have got series of numbers that are sticked to the beta distribution line.

\BeginKnitrBlock{example}[A-R Method for Discrete case]
\protect\hypertarget{exm:ardiscrete}{}{\label{exm:ardiscrete} \iffalse (A-R Method for Discrete case) \fi{} }A-R method can be also implemented to discrete case such as Example \ref{exm:dismass}.
\EndKnitrBlock{example}

\begin{longtable}[]{@{}lrrrrr@{}}
\caption{\label{tab:exdis2}Example of a Discrete Random Variable}\tabularnewline
\toprule
\endhead
x & 0.0 & 1.0 & 2.0 & 3.0 & 4.0\tabularnewline
p & 0.1 & 0.2 & 0.2 & 0.2 & 0.3\tabularnewline
\bottomrule
\end{longtable}

\BeginKnitrBlock{solution}[Generating discrete random numbers using A-R methods]
\iffalse{} {Solution (Generating discrete random numbers using A-R methods). } \fi{}Consider proposal \(g(x) \sim \text{Discrete unif}(0, 1, 2, 3, 4)\), i.e.

\[g(0) = g(1) = \cdots = g(4) = 0.2\]

Then we set

\[c = \max\bigg\{ \frac{p(x)}{g(x)} : x = 0, \ldots, 4 \bigg\} = \max\Big\{ 0.5, 1, 1.5 \Big\} = 1.5\]
\EndKnitrBlock{solution}

\hypertarget{transfomation-methods}{%
\section{Transfomation Methods}\label{transfomation-methods}}

\hypertarget{continuous}{%
\subsection{Continuous}\label{continuous}}

\BeginKnitrBlock{proposition}[Transformation between continuous random variables]
\protect\hypertarget{prp:trans1}{}{\label{prp:trans1} \iffalse (Transformation between continuous random variables) \fi{} }Relation between random variables enables generating target numbers from the others.

\begin{enumerate}
  \item $Z_1, \ldots, Z_n \stackrel{iid}{\sim}N(0, 1) \Rightarrow \sum Z_i^2 \sim \chi^2(n)$
  \item $Y_1 \sim \chi^2(m) \perp\!\!\!\perp Y_2 \sim \chi^2(n) \Rightarrow \frac{Y_1 / m}{Y_2 / n} \sim F(m, n)$
  \item $Z \sim N(0, 1) \perp\!\!\!\perp Y \sim \chi^2(n) \Rightarrow \frac{Z}{\sqrt{Y / n}} \sim t(n)$
  \item $Y_1, \ldots, Y_n \stackrel{iid}{\sim}Exp(\lambda) \Rightarrow \sum Y_i^2 Gamma(n, \lambda)$
  \item $U \sim unif(0, 1) \Rightarrow (b - a)U + a \sim unif(a, b)$
  \item $U \sim Gamma(r, \lambda) \perp\!\!\!\perp V \sim Gamma(s, \lambda) \Rightarrow \frac{U}{U + V} \sim Beta(r, s)$
  \item $Z \sim N(0, 1) \Rightarrow \mu + \sigma Z \sim N(\mu, \sigma^2)$
  \item $Y \sim N(\mu, \sigma^2) \Rightarrow e^Y \sim LogNormal(\mu, \sigma^2)$
\end{enumerate}
\EndKnitrBlock{proposition}

\BeginKnitrBlock{example}[Generating Beta(a, b) using rgamma]
\protect\hypertarget{exm:transbeta}{}{\label{exm:transbeta} \iffalse (Generating Beta(a, b) using rgamma) \fi{} }From Proposition \ref{prp:trans1}, we can generate \(Beta(a,b)\) random numbers using \(Gamma(a, 1)\) and \(Gamma(b, 1)\).
\EndKnitrBlock{example}

\begin{Shaded}
\begin{Highlighting}[]
\NormalTok{trans_beta <-}\StringTok{ }\ControlFlowTok{function}\NormalTok{(n, shape1, shape2) \{}
\NormalTok{  u <-}\StringTok{ }\KeywordTok{rgamma}\NormalTok{(n, }\DataTypeTok{shape =}\NormalTok{ shape1, }\DataTypeTok{rate =} \DecValTok{1}\NormalTok{)}
\NormalTok{  v <-}\StringTok{ }\KeywordTok{rgamma}\NormalTok{(n, }\DataTypeTok{shape =}\NormalTok{ shape2, }\DataTypeTok{rate =} \DecValTok{1}\NormalTok{)}
\NormalTok{  u }\OperatorTok{/}\StringTok{ }\NormalTok{(u }\OperatorTok{+}\StringTok{ }\NormalTok{v)}
\NormalTok{\}}
\end{Highlighting}
\end{Shaded}

\begin{figure}[H]

{\centering \includegraphics[width=0.7\linewidth]{statistical-computing_files/figure-latex/tbetafig-1} 

}

\caption{Beta(3,2) Random numbers from each function, including transformation method}\label{fig:tbetafig}
\end{figure}

\hypertarget{box-muller-transformation}{%
\subsection{Box-Muller transformation}\label{box-muller-transformation}}

Denote that Gaussian cdf has no closed form of \(F_X^{-1}\). Using polar coordiantes, we can generate Normal random numers.

\BeginKnitrBlock{theorem}[Box-Muller transformation]
\protect\hypertarget{thm:bmnorm}{}{\label{thm:bmnorm} \iffalse (Box-Muller transformation) \fi{} }Let \(U_1, U_2 \stackrel{iid}{\sim}unif(0,1)\). Then

\[
\begin{cases}
  Z_1 = \sqrt{-2 \ln U_2} \cos(2\pi U_1) \\
  Z_2 = \sqrt{-2 \ln U_2} \sin(2\pi U_1)
\end{cases}
\]
\EndKnitrBlock{theorem}

\BeginKnitrBlock{proof}
\iffalse{} {Proof. } \fi{}Write

\[
(Z_1, Z_2)^T \sim N\bigg( \begin{bmatrix}
  0 \\
  0
\end{bmatrix}, \begin{bmatrix}
  1 & 0 \\
  0 & 1
\end{bmatrix} \bigg)
\]

Then the joint pdf is given by

\[f_{Z_1, Z_2}(x_1, x_2) = \frac{1}{2\pi}\exp\bigg(-\frac{x_1^2 + x_2^2}{2}\bigg)\]

Consider polar coordiate transformation \((R, \theta)\): \(x_1 = R\cos\theta\) and \(x_2 = R\sin\theta\).

Since it is also random vector,

\begin{equation*}
  \begin{split}
    f_{R, \theta}(r, \theta) & = f_{Z_1, Z_2}(x_1, x_2)\lvert J \rvert \\
    & = \frac{1}{2\pi}\exp\bigg(-\frac{x_1^2 + x_2^2}{2}\bigg)\left\lvert\begin{array}{cc}
      \frac{\partial x_1}{\partial r} & \frac{\partial x_1}{\partial \theta} \\
      \frac{\partial x_2}{\partial r} & \frac{\partial x_2}{\partial \theta}
    \end{array}\right\rvert \\
    & = \frac{1}{2\pi}\exp\bigg(-\frac{r^2}{2}\bigg)\left\lvert\begin{array}{cc}
      \frac{\partial x_1}{\partial r} & \frac{\partial x_1}{\partial \theta} \\
      \frac{\partial x_2}{\partial r} & \frac{\partial x_2}{\partial \theta}
    \end{array}\right\rvert \\
    & = \frac{r}{2\pi}\exp\bigg(-\frac{r^2}{2}\bigg)
  \end{split}
\end{equation*}

Then each marginal density function can be computed as

\begin{equation*}
  \begin{split}
    f_{\theta}(\theta) & = \int_0^\infty \frac{r}{2\pi}\exp\bigg(-\frac{r^2}{2}\bigg) dr \\
    & = \frac{1}{2\pi} I_{(0, 2\pi)}(\theta) \\
    & \stackrel{d}{=} unif(0, 2\pi)
  \end{split}
\end{equation*}

\begin{equation*}
  \begin{split}
    f_R(r) & = \int_0^\theta \frac{r}{2\pi}\exp\bigg(-\frac{r^2}{2}\bigg) d\theta \\
    & = r \exp\bigg(-\frac{r^2}{2}\bigg) I_{(0, \infty)}(r)
  \end{split}
\end{equation*}

Thus,

\[f_{R,\theta} = f_{\theta}f_R \Rightarrow R \perp\!\!\!\perp\theta\]

It follows from inverse transformation theorem that

\[Z_1 = R\cos\theta = \sqrt{-2 \ln U_2} \cos(2\pi U_1)\]

and that

\[Z_2 = R\sin\theta = \sqrt{-2 \ln U_2} \sin(2\pi U_1)\]

where \(U_1, U_2 \stackrel{iid}{\sim}unif(0, 1)\)
\EndKnitrBlock{proof}

\begin{algorithm}[H] \label{alg:algbm}
  \SetAlgoLined
  \SetKwInOut{Input}{input}
  \SetKwInOut{Output}{output}
  \For{$i \leftarrow 1$ \KwTo $n$}{
    $U_1, U_2 \stackrel{iid}{\sim}unif(0,1)$\;
    $z_{2i - 1} = \sqrt{-2\ln U_2}\cos(2\pi U_1)$\;
    $z_{2i} = \sqrt{-2\ln U_2}\sin(2\pi U_1)$\;
  }
  \Output{$z_1, \ldots, z_n \stackrel{iid}{\sim}N(0,1)$}
  \caption{Box-Muller transformation}
\end{algorithm}

\begin{Shaded}
\begin{Highlighting}[]
\NormalTok{bmnorm <-}\StringTok{ }\ControlFlowTok{function}\NormalTok{(n, }\DataTypeTok{mean =} \DecValTok{0}\NormalTok{, }\DataTypeTok{sd =} \DecValTok{1}\NormalTok{) \{}
\NormalTok{  n_bm <-}\StringTok{ }\KeywordTok{ceiling}\NormalTok{(n }\OperatorTok{/}\StringTok{ }\DecValTok{2}\NormalTok{)}
  \KeywordTok{tibble}\NormalTok{(}
    \DataTypeTok{theta =} \KeywordTok{runif}\NormalTok{(}\DataTypeTok{n =}\NormalTok{ n_bm, }\DataTypeTok{max =} \DecValTok{2} \OperatorTok{*}\StringTok{ }\NormalTok{pi),}
    \DataTypeTok{R =} \KeywordTok{sqrt}\NormalTok{(}\OperatorTok{-}\DecValTok{2} \OperatorTok{*}\StringTok{ }\KeywordTok{log}\NormalTok{(}\KeywordTok{runif}\NormalTok{(n_bm)))}
\NormalTok{  ) }\OperatorTok\StringTok{ }
\StringTok{    }\KeywordTok{mutate}\NormalTok{(}
      \DataTypeTok{x1 =}\NormalTok{ R }\OperatorTok{*}\StringTok{ }\KeywordTok{cos}\NormalTok{(theta),}
      \DataTypeTok{x2 =}\NormalTok{ R }\OperatorTok{*}\StringTok{ }\KeywordTok{sin}\NormalTok{(theta)}
\NormalTok{    ) }\OperatorTok\StringTok{ }
\StringTok{    }\KeywordTok{gather}\NormalTok{(x1, x2, }\DataTypeTok{key =} \StringTok{"key"}\NormalTok{, }\DataTypeTok{value =} \StringTok{"value"}\NormalTok{) }\OperatorTok\StringTok{ }
\StringTok{    }\KeywordTok{mutate}\NormalTok{(}\DataTypeTok{value =}\NormalTok{ mean }\OperatorTok{+}\StringTok{ }\NormalTok{sd }\OperatorTok{*}\StringTok{ }\NormalTok{value) }\OperatorTok\StringTok{ }
\StringTok{    }\KeywordTok{select}\NormalTok{(value) }\OperatorTok\StringTok{ }
\StringTok{    }\KeywordTok{pull}\NormalTok{()}
\NormalTok{\}}
\end{Highlighting}
\end{Shaded}

\begin{Shaded}
\begin{Highlighting}[]
\KeywordTok{gg_curve}\NormalTok{(dnorm, }\DataTypeTok{from =} \DecValTok{0}\NormalTok{, }\DataTypeTok{to =} \DecValTok{6}\NormalTok{, }\DataTypeTok{args =} \KeywordTok{list}\NormalTok{(}\DataTypeTok{mean =} \DecValTok{3}\NormalTok{, }\DataTypeTok{sd =} \DecValTok{1}\NormalTok{)) }\OperatorTok{+}
\StringTok{  }\KeywordTok{geom_histogram}\NormalTok{(}
    \DataTypeTok{data =} \KeywordTok{tibble}\NormalTok{(}\DataTypeTok{x =} \KeywordTok{bmnorm}\NormalTok{(}\DecValTok{1000}\NormalTok{, }\DataTypeTok{mean =} \DecValTok{3}\NormalTok{, }\DataTypeTok{sd =} \DecValTok{1}\NormalTok{)),}
    \KeywordTok{aes}\NormalTok{(}\DataTypeTok{x =}\NormalTok{ x, }\DataTypeTok{y =}\NormalTok{ ..density..),}
    \DataTypeTok{bins =} \DecValTok{30}\NormalTok{,}
    \DataTypeTok{fill =} \KeywordTok{gg_hcl}\NormalTok{(}\DecValTok{1}\NormalTok{),}
    \DataTypeTok{alpha =} \FloatTok{.5}
\NormalTok{  )}
\end{Highlighting}
\end{Shaded}

\begin{figure}[H]

{\centering \includegraphics[width=0.7\linewidth]{statistical-computing_files/figure-latex/boxnorm-1} 

}

\caption{Normal random numbers by Box-Muller transformation}\label{fig:boxnorm}
\end{figure}

\hypertarget{discrete}{%
\subsection{Discrete}\label{discrete}}

\BeginKnitrBlock{proposition}[Transformation between discrete random variables]
\protect\hypertarget{prp:trans2}{}{\label{prp:trans2} \iffalse (Transformation between discrete random variables) \fi{} }Relation between random variables enables generating target numbers from the others.

\begin{enumerate}
  \item $Y_1, \ldots, Y_n \stackrel{iid}{\sim}Bernoulli(p) \Rightarrow \sum Y_i^2 \sim B(n, p)$
  \item $U \sim unif(0,1) \Rightarrow X_i = \lfloor mU \rfloor + 1$
  \item $X = \text{the number of events occurring in 1 unit of time} \sim Poisson(\lambda)$
\end{enumerate}
\EndKnitrBlock{proposition}

\BeginKnitrBlock{proposition}[Bernoulli process]
\protect\hypertarget{prp:trans3}{}{\label{prp:trans3} \iffalse (Bernoulli process) \fi{} }Let \(X_1, X_2, \ldots \stackrel{iid}{\sim}Bernoulli(p)\).

\begin{enumerate}
  \item $N = \text{the number of trials until we see a success, i.e.} X_N = 1 \Rightarrow N \sim Geo(p)$
  \item $Y_1, \ldots, Y_r \stackrel{iid}{\sim}Geo(p) \Rightarrow \sum\limits_{i = 1}^r Y_i = \text{the number of trials until we see r successes} \sim NegBin(r, p)$
\end{enumerate}
\EndKnitrBlock{proposition}

\BeginKnitrBlock{proposition}[Count process]
\protect\hypertarget{prp:trans4}{}{\label{prp:trans4} \iffalse (Count process) \fi{} }Let \(Y_1, Y_2, \ldots \stackrel{iid}{\sim}Exp(\lambda)\) be interarrival times. Then

\[X = \max\{ n : \sum Y_i \le 1 \} = \text{the number of events occurring in 1 unit of time} \sim Poisson(\lambda)\]
\EndKnitrBlock{proposition}

\hypertarget{sums-and-mixtures}{%
\section{Sums and Mixtures}\label{sums-and-mixtures}}

\hypertarget{convolutions}{%
\subsection{Convolutions}\label{convolutions}}

\BeginKnitrBlock{definition}[Convolution]
\protect\hypertarget{def:conv}{}{\label{def:conv} \iffalse (Convolution) \fi{} }Let \(X_1, \ldots, X_n\) be independent and identically distributed and let \(S = X_1 + \cdots X_n\). Then the distribution of \(S\) is called the \(n\)-fold convolution of \(X\) and denoted by \(F_X^{*(n)}\).
\EndKnitrBlock{definition}

In the last chapter, we have already seen a bunch of random variables that can be generated by summing the other.

\BeginKnitrBlock{example}[Chisquare]
\protect\hypertarget{exm:rchi}{}{\label{exm:rchi} \iffalse (Chisquare) \fi{} }Let \(Z_1, \ldots, Z_n \stackrel{iid}{\sim}N(0, 1)\). We know from Proposition \ref{prp:trans1} that

\[V = \sum_{i = 1}^n Z_i \sim \chi^2(n)\]
\EndKnitrBlock{example}

Building a \texttt{n} \(\times\) \texttt{df} matrix can be a good strategy here. After that, \texttt{rowSums} or \texttt{colSums} ends the generation work.

\begin{Shaded}
\begin{Highlighting}[]
\NormalTok{conv_chisq <-}\StringTok{ }\ControlFlowTok{function}\NormalTok{(n, df) \{}
\NormalTok{  X <-}
\StringTok{    }\KeywordTok{matrix}\NormalTok{(}\KeywordTok{rnorm}\NormalTok{(n }\OperatorTok{*}\StringTok{ }\NormalTok{df), }\DataTypeTok{nrow =}\NormalTok{ n, }\DataTypeTok{ncol =}\NormalTok{ df)}\OperatorTok{^}\DecValTok{2}
  \KeywordTok{rowSums}\NormalTok{(X)}
\NormalTok{\}}
\end{Highlighting}
\end{Shaded}

\begin{Shaded}
\begin{Highlighting}[]
\KeywordTok{gg_curve}\NormalTok{(dchisq, }\DataTypeTok{from =} \DecValTok{0}\NormalTok{, }\DataTypeTok{to =} \DecValTok{15}\NormalTok{, }\DataTypeTok{args =} \KeywordTok{list}\NormalTok{(}\DataTypeTok{df =} \DecValTok{5}\NormalTok{)) }\OperatorTok{+}
\StringTok{  }\KeywordTok{geom_histogram}\NormalTok{(}
    \DataTypeTok{data =} \KeywordTok{tibble}\NormalTok{(}\DataTypeTok{x =} \KeywordTok{conv_chisq}\NormalTok{(}\DecValTok{1000}\NormalTok{, }\DataTypeTok{df =} \DecValTok{5}\NormalTok{)),}
    \KeywordTok{aes}\NormalTok{(}\DataTypeTok{x =}\NormalTok{ x, }\DataTypeTok{y =}\NormalTok{ ..density..),}
    \DataTypeTok{bins =} \DecValTok{30}\NormalTok{,}
    \DataTypeTok{fill =} \KeywordTok{gg_hcl}\NormalTok{(}\DecValTok{1}\NormalTok{),}
    \DataTypeTok{alpha =} \FloatTok{.5}
\NormalTok{  )}
\end{Highlighting}
\end{Shaded}

\begin{figure}[H]

{\centering \includegraphics[width=0.7\linewidth]{statistical-computing_files/figure-latex/convchi-1} 

}

\caption{$\chi^2$ random numbers from Normal sums}\label{fig:convchi}
\end{figure}

\hypertarget{mixtures}{%
\subsection{Mixtures}\label{mixtures}}

\BeginKnitrBlock{definition}[Discrete mixture]
\protect\hypertarget{def:mixprob1}{}{\label{def:mixprob1} \iffalse (Discrete mixture) \fi{} }A random variable \(X\) is a discrete mixture if the distribution of \(X\) is a weighted sum

\[F_X(x) = \sum \theta_i F_{X_i}(x)\]

where constants \(\theta_i\) are called the mixing weights or mixing probabilities.
\EndKnitrBlock{definition}

\BeginKnitrBlock{definition}[Continuous mixture]
\protect\hypertarget{def:mixprob2}{}{\label{def:mixprob2} \iffalse (Continuous mixture) \fi{} }A random variable \(X\) is a continuous mixture if the distribution of \(X\) is a weighted sum

\[F_X(x) = \int_\infty^\infty F_{X \mid Y = y} (x) f_Y(y) dy\]
\EndKnitrBlock{definition}

\BeginKnitrBlock{example}[Mixture of several Normal distributions]
\protect\hypertarget{exm:gaussmix}{}{\label{exm:gaussmix} \iffalse (Mixture of several Normal distributions) \fi{} }Generate a random sample of size \(1000\) from a normal location mixture with components of the mixture \(N(0,1)\) and \(N(3,1)\), i.e.

\[F_X = p_1 F_{X_1} + (1 - p_1) F_{X_2}\]
\EndKnitrBlock{example}

For easy combining samples, we use \texttt{foreach} library.

\begin{Shaded}
\begin{Highlighting}[]
\KeywordTok{library}\NormalTok{(foreach)}
\end{Highlighting}
\end{Shaded}

As in A-R method, Bernoullin splitting would be used.

\[
\begin{cases}
  F_{X_1} & U > p_1 \\
  F_{X_2} & \text{otherwise}
\end{cases}
\]

\begin{Shaded}
\begin{Highlighting}[]
\NormalTok{mix_norm <-}\StringTok{ }\ControlFlowTok{function}\NormalTok{(n, p1, mean1, sd1, mean2, sd2) \{}
\NormalTok{  x1 <-}\StringTok{ }\KeywordTok{rnorm}\NormalTok{(n, }\DataTypeTok{mean =}\NormalTok{ mean1, }\DataTypeTok{sd =}\NormalTok{ sd1)}
\NormalTok{  x2 <-}\StringTok{ }\KeywordTok{rnorm}\NormalTok{(n, }\DataTypeTok{mean =}\NormalTok{ mean2, }\DataTypeTok{sd =}\NormalTok{ sd2)}
\NormalTok{  k <-}\StringTok{ }\KeywordTok{as.integer}\NormalTok{(}\KeywordTok{runif}\NormalTok{(n) }\OperatorTok{>}\StringTok{ }\NormalTok{p1)}
\NormalTok{  k }\OperatorTok{*}\StringTok{ }\NormalTok{x1 }\OperatorTok{+}\StringTok{ }\NormalTok{(}\DecValTok{1} \OperatorTok{-}\StringTok{ }\NormalTok{k) }\OperatorTok{*}\StringTok{ }\NormalTok{x2}
\NormalTok{\}}
\end{Highlighting}
\end{Shaded}

Try various \(p_1\), from 0.1 to 1

\begin{Shaded}
\begin{Highlighting}[]
\NormalTok{mixture <-}
\StringTok{  }\KeywordTok{foreach}\NormalTok{(}\DataTypeTok{p1 =} \DecValTok{0}\OperatorTok{:}\DecValTok{10} \OperatorTok{/}\StringTok{ }\DecValTok{10}\NormalTok{, }\DataTypeTok{.combine =}\NormalTok{ bind_rows) }\OperatorTok\StringTok{ }\NormalTok{\{}
    \KeywordTok{tibble}\NormalTok{(}
      \DataTypeTok{value =} \KeywordTok{mix_norm}\NormalTok{(}\DataTypeTok{n =} \DecValTok{1000}\NormalTok{, }\DataTypeTok{p1 =}\NormalTok{ p1, }\DataTypeTok{mean1 =} \DecValTok{0}\NormalTok{, }\DataTypeTok{sd1 =} \DecValTok{1}\NormalTok{, }\DataTypeTok{mean2 =} \DecValTok{3}\NormalTok{, }\DataTypeTok{sd2 =} \DecValTok{1}\NormalTok{),}
      \DataTypeTok{key =} \KeywordTok{rep}\NormalTok{(p1, }\DecValTok{1000}\NormalTok{)}
\NormalTok{    )}
\NormalTok{  \}}
\end{Highlighting}
\end{Shaded}

\begin{Shaded}
\begin{Highlighting}[]
\NormalTok{mixture }\OperatorTok\StringTok{ }
\StringTok{  }\KeywordTok{ggplot}\NormalTok{(}\KeywordTok{aes}\NormalTok{(}\DataTypeTok{x =}\NormalTok{ value, }\DataTypeTok{colour =} \KeywordTok{factor}\NormalTok{(key))) }\OperatorTok{+}
\StringTok{  }\KeywordTok{stat_density}\NormalTok{(}\DataTypeTok{geom =} \StringTok{"line"}\NormalTok{, }\DataTypeTok{position =} \StringTok{"identity"}\NormalTok{) }\OperatorTok{+}
\StringTok{  }\KeywordTok{scale_colour_discrete}\NormalTok{(}
    \DataTypeTok{name =} \KeywordTok{expression}\NormalTok{(p[}\DecValTok{1}\NormalTok{]),}
    \DataTypeTok{labels =} \DecValTok{0}\OperatorTok{:}\DecValTok{10} \OperatorTok{/}\StringTok{ }\DecValTok{10}
\NormalTok{  ) }\OperatorTok{+}
\StringTok{  }\KeywordTok{xlab}\NormalTok{(}\StringTok{"x"}\NormalTok{)}
\end{Highlighting}
\end{Shaded}

\begin{figure}[H]

{\centering \includegraphics[width=0.7\linewidth]{statistical-computing_files/figure-latex/bimodal-1} 

}

\caption{Mixture normal random number for each mixing probability}\label{fig:bimodal}
\end{figure}

\hypertarget{multivariate-normal-random-vector}{%
\section{Multivariate Normal Random Vector}\label{multivariate-normal-random-vector}}

\BeginKnitrBlock{definition}[Multivariate normal random vector]
\protect\hypertarget{def:mvn}{}{\label{def:mvn} \iffalse (Multivariate normal random vector) \fi{} }A random vector \(\mathbf{X} = (X_1, \ldots, X_p)^T\) follows multivariate normal distribution if

\[f(\mathbf{x}) = \frac{1}{(2\pi)^{\frac{p}{2}\lvert \Sigma \rvert}} \exp \bigg[ -\frac{1}{2}(\mathbf{x} \boldsymbol\mu)^T \Sigma^{-1}(\mathbf{x} \boldsymbol\mu) \bigg]\]
\EndKnitrBlock{definition}

\BeginKnitrBlock{remark}
\iffalse{} {Remark. } \fi{}Let \(\mathbf{Z} \sim MVN(\mathbf{0}, I)\). Then

\begin{equation}
  \Sigma^{\frac{1}{2}}\mathbf{Z} + \boldsymbol\mu \sim MVN(\boldsymbol\mu, \Sigma)
  \label{eq:stdmvn}
\end{equation}
\EndKnitrBlock{remark}

From this remark, we get to generate \emph{standard normal random vector}.

\hypertarget{spectral-decomposition-method}{%
\subsection{Spectral decomposition method}\label{spectral-decomposition-method}}

Note that covariance matrix is symmetric.

\BeginKnitrBlock{theorem}[Spectral decomposition]
\protect\hypertarget{thm:covspec}{}{\label{thm:covspec} \iffalse (Spectral decomposition) \fi{} }Suppose that \(\Sigma\) is symmetric. Then

\[\Sigma = P \Lambda P^T\]

where \((\mathbf{v}_j, \lambda_j)\) corresponding eigenvector-eigenvalue

\[
\begin{cases}
  P = \begin{bmatrix} \mathbf{v}_1 & \cdots & \mathbf{v}_p \end{bmatrix} \in \mathbb{R}^{p \times p} \:\text{orthogonal} \\
  \Lambda = diag(\lambda_1, \ldots, \lambda_p)
\end{cases}
\]
\EndKnitrBlock{theorem}

\BeginKnitrBlock{corollary}
\protect\hypertarget{cor:specsqrt}{}{\label{cor:specsqrt} }Suppose that \(\Sigma\) is symmetric. Then

\[\Sigma^{\frac{1}{2}} = P \Lambda^{\frac{1}{2}} P^T\]

where \(\Lambda^{\frac{1}{2}} = diag(\sqrt{\lambda_1}, \ldots, \sqrt{\lambda_p})\)
\EndKnitrBlock{corollary}

\texttt{eigen()} performs spectral decomposition. \texttt{\$values} has eigenvalues and \texttt{\$vectors} has eigenvectors. We first generate matrix that consists of standard normal random vector:

\[
\begin{bmatrix}
  Z_{11} & Z_{12} & \cdots & Z_{1p} \\
  Z_{21} & Z_{22} & \cdots & Z_{2p} \\
  \vdots & \vdots & \vdots & \vdots \\
  Z_{n1} & Z_{n2} & \cdots & Z_{np}
\end{bmatrix}
\]

Denote that each observation is row. To use Equation \eqref{eq:stdmvn}, we should multiply \(\Sigma^{\frac{1}{2}}\) behind this matrix, not in front of. \(\boldsymbol\mu\) matrix should be also made to matrix, in form of

\[
\begin{bmatrix}
  \mu_{11} & \mu_{12} & \cdots & \mu_{1p} \\
  \mu_{11} & \mu_{22} & \cdots & \mu_{1p} \\
  \vdots & \vdots & \vdots & \vdots \\
  \mu_{11} & Z_{n2} & \cdots & \mu_{1p}
\end{bmatrix} \in \mathbb{R}^{n \times p}
\]

\begin{Shaded}
\begin{Highlighting}[]
\NormalTok{rmvn_eigen <-}\StringTok{ }\ControlFlowTok{function}\NormalTok{(n, mu, sig) \{}
\NormalTok{  d <-}\StringTok{ }\KeywordTok{length}\NormalTok{(mu)}
\NormalTok{  ev <-}\StringTok{ }\KeywordTok{eigen}\NormalTok{(sig, }\DataTypeTok{symmetric =} \OtherTok{TRUE}\NormalTok{)}
\NormalTok{  lambda <-}\StringTok{ }\NormalTok{ev}\OperatorTok{$}\NormalTok{values}
\NormalTok{  P <-}\StringTok{ }\NormalTok{ev}\OperatorTok{$}\NormalTok{vectors}
\NormalTok{  sig2 <-}\StringTok{ }\NormalTok{P }\OperatorTok\StringTok{ }\KeywordTok{diag}\NormalTok{(}\KeywordTok{sqrt}\NormalTok{(lambda)) }\OperatorTok\StringTok{ }\KeywordTok{t}\NormalTok{(P)}
\NormalTok{  Z <-}\StringTok{ }\KeywordTok{matrix}\NormalTok{(}\KeywordTok{rnorm}\NormalTok{(n }\OperatorTok{*}\StringTok{ }\NormalTok{d), }\DataTypeTok{nrow =}\NormalTok{ n, }\DataTypeTok{ncol =}\NormalTok{ d)}
\NormalTok{  X <-}\StringTok{ }\NormalTok{Z }\OperatorTok\StringTok{ }\NormalTok{sig2 }\OperatorTok{+}\StringTok{ }\KeywordTok{matrix}\NormalTok{(mu, }\DataTypeTok{nrow =}\NormalTok{ n, }\DataTypeTok{ncol =}\NormalTok{ d, }\DataTypeTok{byrow =} \OtherTok{TRUE}\NormalTok{)}
  \KeywordTok{colnames}\NormalTok{(X) <-}\StringTok{ }\KeywordTok{paste0}\NormalTok{(}\StringTok{"x"}\NormalTok{, }\DecValTok{1}\OperatorTok{:}\NormalTok{d)}
\NormalTok{  X }\OperatorTok\StringTok{ }\KeywordTok{tbl_df}\NormalTok{()}
\NormalTok{\}}
\end{Highlighting}
\end{Shaded}

\begin{Shaded}
\begin{Highlighting}[]
\CommentTok{# mean vector -------------------------------}
\NormalTok{mu <-}\StringTok{ }\KeywordTok{c}\NormalTok{(}\DecValTok{0}\NormalTok{, }\DecValTok{1}\NormalTok{, }\DecValTok{2}\NormalTok{)}
\CommentTok{# symmetric matrix --------------------------}
\NormalTok{sig <-}\StringTok{ }\KeywordTok{matrix}\NormalTok{(}\KeywordTok{numeric}\NormalTok{(}\DecValTok{9}\NormalTok{), }\DataTypeTok{nrow =} \DecValTok{3}\NormalTok{, }\DataTypeTok{ncol =} \DecValTok{3}\NormalTok{)}
\KeywordTok{diag}\NormalTok{(sig) <-}\StringTok{ }\KeywordTok{rep}\NormalTok{(}\DecValTok{1}\NormalTok{, }\DecValTok{3}\NormalTok{)}
\NormalTok{sig[}\KeywordTok{lower.tri}\NormalTok{(sig)] <-}\StringTok{ }\KeywordTok{c}\NormalTok{(}\OperatorTok{-}\NormalTok{.}\DecValTok{5}\NormalTok{, }\FloatTok{.5}\NormalTok{, }\FloatTok{-.5}\NormalTok{) }\OperatorTok{*}\StringTok{ }\DecValTok{2}
\NormalTok{sig <-}\StringTok{ }\NormalTok{(sig }\OperatorTok{+}\StringTok{ }\KeywordTok{t}\NormalTok{(sig)) }\OperatorTok{/}\StringTok{ }\DecValTok{2}
\end{Highlighting}
\end{Shaded}

Generate

\[\mathbf{X}_i \sim MVN\bigg((0, 1, 2), \begin{bmatrix} 1&-0.5&0.5 \\ -0.5&1&-0.5 \\ 0.5&-0.5&1 \\ \end{bmatrix}\bigg)\]

\begin{Shaded}
\begin{Highlighting}[]
\NormalTok{(mvn3 <-}\StringTok{ }\KeywordTok{rmvn_eigen}\NormalTok{(}\DecValTok{1000}\NormalTok{, }\DataTypeTok{mu =}\NormalTok{ mu, }\DataTypeTok{sig =}\NormalTok{ sig))}
\CommentTok{#> # A tibble: 1,000 x 3}
\CommentTok{#>         x1       x2    x3}
\CommentTok{#>      <dbl>    <dbl> <dbl>}
\CommentTok{#>  1 -0.168   1.41    1.80 }
\CommentTok{#>  2  1.39   -0.00942 2.40 }
\CommentTok{#>  3 -0.710   1.30    1.37 }
\CommentTok{#>  4  0.0314  2.04    1.80 }
\CommentTok{#>  5  0.177   0.568   1.71 }
\CommentTok{#>  6 -0.960   1.23    1.61 }
\CommentTok{#>  7 -1.01    1.28    0.106}
\CommentTok{#>  8  0.272   0.0842  2.12 }
\CommentTok{#>  9  0.148   1.63    2.53 }
\CommentTok{#> 10 -1.24    1.53    1.28 }
\CommentTok{#> # ... with 990 more rows}
\end{Highlighting}
\end{Shaded}

\begin{Shaded}
\begin{Highlighting}[]
\NormalTok{mvn3 }\OperatorTok\StringTok{ }
\StringTok{  }\NormalTok{GGally}\OperatorTok{::}\KeywordTok{ggpairs}\NormalTok{(}
    \DataTypeTok{lower =} \KeywordTok{list}\NormalTok{(}\DataTypeTok{continuous =}\NormalTok{ GGally}\OperatorTok{::}\KeywordTok{wrap}\NormalTok{(gg_scatter, }\DataTypeTok{size =} \DecValTok{1}\NormalTok{))}
\NormalTok{  )}
\end{Highlighting}
\end{Shaded}

\begin{figure}[H]

{\centering \includegraphics[width=0.7\linewidth]{statistical-computing_files/figure-latex/mvneigen-1} 

}

\caption{Multivariate normal random vector - spectral decomposition method}\label{fig:mvneigen}
\end{figure}

\hypertarget{singular-value-decomposition}{%
\subsection{Singular value decomposition}\label{singular-value-decomposition}}

SVD can be said to be a kind of generalization of spectral decomposition. This method can be used for any matrix, i.e.~non-symmetric matrix. For \(\Sigma\), SVD and spectral decomposition is equivalent. However, SVD does not account for symmetric property, so this method is less efficient compared to spectral decomposition.

\begin{Shaded}
\begin{Highlighting}[]
\NormalTok{rmvn_svd <-}\StringTok{ }\ControlFlowTok{function}\NormalTok{(n, mu, sig) \{}
\NormalTok{  d <-}\StringTok{ }\KeywordTok{length}\NormalTok{(mu)}
\NormalTok{  S <-}\StringTok{ }\KeywordTok{svd}\NormalTok{(sig)}
\NormalTok{  sig2 <-}\StringTok{ }\NormalTok{S}\OperatorTok{$}\NormalTok{u }\OperatorTok\StringTok{ }\KeywordTok{diag}\NormalTok{(}\KeywordTok{sqrt}\NormalTok{(S}\OperatorTok{$}\NormalTok{d)) }\OperatorTok\StringTok{ }\KeywordTok{t}\NormalTok{(S}\OperatorTok{$}\NormalTok{v)}
\NormalTok{  Z <-}\StringTok{ }\KeywordTok{matrix}\NormalTok{(}\KeywordTok{rnorm}\NormalTok{(n }\OperatorTok{*}\StringTok{ }\NormalTok{d), }\DataTypeTok{nrow =}\NormalTok{ n, }\DataTypeTok{ncol =}\NormalTok{ d)}
\NormalTok{  X <-}\StringTok{ }\NormalTok{Z }\OperatorTok\StringTok{ }\NormalTok{sig2 }\OperatorTok{+}\StringTok{ }\KeywordTok{matrix}\NormalTok{(mu, }\DataTypeTok{nrow =}\NormalTok{ n, }\DataTypeTok{ncol =}\NormalTok{ d, }\DataTypeTok{byrow =} \OtherTok{TRUE}\NormalTok{)}
  \KeywordTok{colnames}\NormalTok{(X) <-}\StringTok{ }\KeywordTok{paste0}\NormalTok{(}\StringTok{"x"}\NormalTok{, }\DecValTok{1}\OperatorTok{:}\NormalTok{d)}
\NormalTok{  X }\OperatorTok\StringTok{ }\KeywordTok{tbl_df}\NormalTok{()}
\NormalTok{\}}
\end{Highlighting}
\end{Shaded}

\begin{figure}[H]

{\centering \includegraphics[width=0.7\linewidth]{statistical-computing_files/figure-latex/mvnsvd-1} 

}

\caption{Multivariate normal random vector - svd}\label{fig:mvnsvd}
\end{figure}

\hypertarget{choleski-decomposition}{%
\subsection{Choleski decomposition}\label{choleski-decomposition}}

\BeginKnitrBlock{theorem}[Cholesky decomposition]
\protect\hypertarget{thm:covchol}{}{\label{thm:covchol} \iffalse (Cholesky decomposition) \fi{} }Suppose that \(\Sigma\) is symmetric and positive definite. Then

\[\Sigma = Q^T Q\]

where \(Q\) is an upper triangular matrix.
\EndKnitrBlock{theorem}

\BeginKnitrBlock{corollary}
\protect\hypertarget{cor:mvnchol2}{}{\label{cor:mvnchol2} }Suppose that \(\Sigma\) is symmetric and positive definite. For cholesky decomposition \ref{thm:covchol}, define

\[\Sigma^{\frac{1}{2}} = Q\]
\EndKnitrBlock{corollary}

\texttt{chol()} computes cholesky decomposition. In \texttt{R}, it gives upper triangular \(Q\). Since some statements cholesky decomposition by \(\Sigma = LL^T\) with lower triangular matrix, try not to confuse.

\begin{Shaded}
\begin{Highlighting}[]
\NormalTok{rmvn_chol <-}\StringTok{ }\ControlFlowTok{function}\NormalTok{(n, mu, sig) \{}
\NormalTok{  d <-}\StringTok{ }\KeywordTok{length}\NormalTok{(mu)}
\NormalTok{  sig2 <-}\StringTok{ }\KeywordTok{chol}\NormalTok{(sig)}
\NormalTok{  Z <-}\StringTok{ }\KeywordTok{matrix}\NormalTok{(}\KeywordTok{rnorm}\NormalTok{(n }\OperatorTok{*}\StringTok{ }\NormalTok{d), }\DataTypeTok{nrow =}\NormalTok{ n, }\DataTypeTok{ncol =}\NormalTok{ d)}
\NormalTok{  X <-}\StringTok{ }\NormalTok{Z }\OperatorTok\StringTok{ }\NormalTok{sig2 }\OperatorTok{+}\StringTok{ }\KeywordTok{matrix}\NormalTok{(mu, }\DataTypeTok{nrow =}\NormalTok{ n, }\DataTypeTok{ncol =}\NormalTok{ d, }\DataTypeTok{byrow =} \OtherTok{TRUE}\NormalTok{)}
  \KeywordTok{colnames}\NormalTok{(X) <-}\StringTok{ }\KeywordTok{paste0}\NormalTok{(}\StringTok{"x"}\NormalTok{, }\DecValTok{1}\OperatorTok{:}\NormalTok{d)}
\NormalTok{  X }\OperatorTok\StringTok{ }\KeywordTok{tbl_df}\NormalTok{()}
\NormalTok{\}}
\end{Highlighting}
\end{Shaded}

\begin{figure}[H]

{\centering \includegraphics[width=0.7\linewidth]{statistical-computing_files/figure-latex/mvnchol-1} 

}

\caption{Multivariate normal random vector - cholesky decomposition}\label{fig:mvnchol}
\end{figure}

\hypertarget{stochastic-processes}{%
\section{Stochastic Processes}\label{stochastic-processes}}

\BeginKnitrBlock{definition}[Stochastic process]
\protect\hypertarget{def:stoc}{}{\label{def:stoc} \iffalse (Stochastic process) \fi{} }A stochastic process is a collection \(\{ X(t) : t \in T \}\) of random variables indexed by the set \(T\). The index set \(T\) could be discrete or continous.

A State space is called te set of possible values that \(X(t)\) can take.
\EndKnitrBlock{definition}

\hypertarget{homogeneous-poisson-process}{%
\subsection{Homogeneous poisson process}\label{homogeneous-poisson-process}}

\hypertarget{nonhomogeneous-poisson-process}{%
\subsection{Nonhomogeneous poisson process}\label{nonhomogeneous-poisson-process}}

\hypertarget{symmetric-random-walk}{%
\subsection{Symmetric random walk}\label{symmetric-random-walk}}

\hypertarget{mcint}{%
\chapter{Monte Carlo Integration and Variance Reduction}\label{mcint}}

\hypertarget{monte-carlo-integration}{%
\section{Monte Carlo Integration}\label{monte-carlo-integration}}

Consider integration problem of a integrable function \(g(x)\). We want to compute

\[\theta \equiv \int_a^b g(x) dx\]

For instance, standard normal cdf.

\BeginKnitrBlock{example}[Standard normal cdf]
\protect\hypertarget{exm:mcex}{}{\label{exm:mcex} \iffalse (Standard normal cdf) \fi{} }Compute values for

\[\Phi(x) = \int_{-\infty}^x \frac{1}{\sqrt{2\pi}}\exp\bigg(-\frac{t^2}{2}\bigg)dt\]
\EndKnitrBlock{example}

It might be impossible to compute this integral with hand. So we implement \emph{simulation} concept here, based on the following theorems.

\BeginKnitrBlock{theorem}[Weak Law of Large Numbers]
\protect\hypertarget{thm:wlaw}{}{\label{thm:wlaw} \iffalse (Weak Law of Large Numbers) \fi{} }Suppose that \(X_1, \ldots, X_n \stackrel{iid}{\sim}(\mu, \sigma^2 < \infty)\). Then

\[\frac{1}{n}\sum_{i = 1}^n X_i \stackrel{p}{\rightarrow} \mu\]

Let \(g\) be a measurable function. Then

\[\frac{1}{n}\sum_{i = 1}^n g(X_i) \stackrel{p}{\rightarrow} g(\mu)\]
\EndKnitrBlock{theorem}

\BeginKnitrBlock{theorem}[Strong Law of Large Numbers]
\protect\hypertarget{thm:slaw}{}{\label{thm:slaw} \iffalse (Strong Law of Large Numbers) \fi{} }Suppose that \(X_1, \ldots, X_n \stackrel{iid}{\sim}(\mu, \sigma^2 < \infty)\). Then

\[\frac{1}{n}\sum_{i = 1}^n X_i \stackrel{a.s.}{\rightarrow} \mu\]

Let \(g\) be a measurable function. Then

\[\frac{1}{n}\sum_{i = 1}^n g(X_i) \stackrel{a.s.}{\rightarrow} g(\mu)\]
\EndKnitrBlock{theorem}

\hypertarget{simple-monte-carlo-estimator}{%
\subsection{Simple Monte Carlo estimator}\label{simple-monte-carlo-estimator}}

\BeginKnitrBlock{theorem}[Monte Carlo Integration]
\protect\hypertarget{thm:mcint}{}{\label{thm:mcint} \iffalse (Monte Carlo Integration) \fi{} }Consider integration \eqref{eq:muint}. This can be approximated via appropriate pdf \(f(x)\) by

\[\hat\theta_M = \frac{1}{N} \sum_{i = 1}^N g(X_i)\]
\EndKnitrBlock{theorem}

Suppose that we have a distribution \(f(x) = I_{spt g}(x)\), i.e.~\emph{uniform distribution}. Let \(spt g = (a, b)\).

\begin{equation}
  \begin{split}
    \theta & \equiv \int_{spt g} g(x) dx \\
    & = \int_a^b g(x) dx \\
    & = \int_0^1 g(a + (b - a)t)(b - a) dt \\
    & \equiv \int_0^1 h(t) dt \\
    & = \int_0^1 h(t) I_{(a, b)}(t) dt \\
    & = E[h(U)] \qquad U \sim unif(0, 1)
  \end{split}
  \label{eq:muint}
\end{equation}

By \emph{the Strong law of large numbers} \ref{thm:slaw},

\[\frac{1}{n}\sum_{i = 1}^n h(U_i) \stackrel{a.s.}{\rightarrow} E\Big[h(U)\Big] = \theta\]

where \(U \sim unif(0, 1)\). Thus, what we have to do here are two things.

\begin{enumerate}
\def\labelenumi{\arabic{enumi}.}
\tightlist
\item
  representing \(g\) as \(h\).
\item
  generating lots of \(U_i\)
\end{enumerate}

Go back to Example \ref{exm:mcex}.

\BeginKnitrBlock{solution}
\iffalse{} {Solution. } \fi{}Case 1: \(x > 0\)

Since \(\Phi(x)\) is symmetry,

\[\Phi(0) = \frac{1}{2}\]

Fix \(x > 0\).

\begin{equation*}
  \begin{split}
    \int_0^x \exp\bigg(-\frac{t^2}{2}\bigg) dt & = \int_0^x x\exp\bigg(-\frac{t^2}{2}\bigg)\frac{I_{(0, x)}(t)}{x} dt \\
    & \approx \frac{1}{N} \sum_{i = 1}^N x\exp\bigg(-\frac{U_i^2}{2}\bigg)
  \end{split}
\end{equation*}

with \(U_1, \ldots, U_N \stackrel{iid}{\sim}unif(0, x)\).

Case 2: \(x \le 0\)

Recall that \(\Phi(x)\) is symmetry.

Hence,

\[
\hat\Phi(x) = \begin{cases}
  \frac{1}{\sqrt{2\pi}} \frac{1}{N} \sum_{i = 1}^N x\exp\bigg(-\frac{U_i^2}{2}\bigg) + \frac{1}{2} \equiv \hat\theta(x) & x \ge 0 \\
  1 - \hat\theta(-x) & x < 0
\end{cases}
\]
\EndKnitrBlock{solution}

\begin{Shaded}
\begin{Highlighting}[]
\NormalTok{phihat <-}\StringTok{ }\ControlFlowTok{function}\NormalTok{(x, y) \{}
\NormalTok{  yi <-}\StringTok{ }\KeywordTok{abs}\NormalTok{(y)}
\NormalTok{  theta <-}\StringTok{ }\KeywordTok{mean}\NormalTok{(yi }\OperatorTok{*}\StringTok{ }\KeywordTok{exp}\NormalTok{(}\OperatorTok{-}\NormalTok{x}\OperatorTok{^}\DecValTok{2} \OperatorTok{/}\StringTok{ }\DecValTok{2}\NormalTok{)) }\OperatorTok{/}\StringTok{ }\KeywordTok{sqrt}\NormalTok{(}\DecValTok{2} \OperatorTok{*}\StringTok{ }\NormalTok{pi) }\OperatorTok{+}\StringTok{ }\FloatTok{.5}
  \KeywordTok{ifelse}\NormalTok{(y }\OperatorTok{>=}\StringTok{ }\DecValTok{0}\NormalTok{, theta, }\DecValTok{1} \OperatorTok{-}\StringTok{ }\NormalTok{theta)}
\NormalTok{\}}
\end{Highlighting}
\end{Shaded}

Then compute \(\hat\Phi(x)\) for various \(x\) values.

\begin{Shaded}
\begin{Highlighting}[]
\NormalTok{phi_simul <-}\StringTok{ }\KeywordTok{foreach}\NormalTok{(}\DataTypeTok{y =} \KeywordTok{seq}\NormalTok{(.}\DecValTok{1}\NormalTok{, }\FloatTok{2.5}\NormalTok{, }\DataTypeTok{length.out =} \DecValTok{10}\NormalTok{), }\DataTypeTok{.combine =}\NormalTok{ bind_rows) }\OperatorTok\StringTok{ }\NormalTok{\{}
  \KeywordTok{tibble}\NormalTok{(}
    \DataTypeTok{x =}\NormalTok{ y,}
    \DataTypeTok{phi =} \KeywordTok{pnorm}\NormalTok{(y),}
    \DataTypeTok{Phihat =} 
      \KeywordTok{tibble}\NormalTok{(}\DataTypeTok{x =} \KeywordTok{runif}\NormalTok{(}\DecValTok{10000}\NormalTok{, }\DataTypeTok{max =}\NormalTok{ y)) }\OperatorTok\StringTok{ }
\StringTok{      }\KeywordTok{summarise}\NormalTok{(}\DataTypeTok{cdf =} \KeywordTok{phihat}\NormalTok{(x, }\DataTypeTok{y =}\NormalTok{ y)) }\OperatorTok\StringTok{ }
\StringTok{      }\KeywordTok{pull}\NormalTok{()}
\NormalTok{  )}
\NormalTok{\}}
\end{Highlighting}
\end{Shaded}

\begin{longtable}{r|r|r}
\caption{\label{tab:unnamed-chunk-35}Simple MC estimates of Normal cdf for each x}\\
\hline
x & pnorm & mc\\
\hline
0.100 & 0.540 & 0.540\\
\hline
0.367 & 0.643 & 0.643\\
\hline
0.633 & 0.737 & 0.737\\
\hline
0.900 & 0.816 & 0.816\\
\hline
1.167 & 0.878 & 0.878\\
\hline
1.433 & 0.924 & 0.923\\
\hline
1.700 & 0.955 & 0.958\\
\hline
1.967 & 0.975 & 0.976\\
\hline
2.233 & 0.987 & 0.987\\
\hline
2.500 & 0.994 & 0.990\\
\hline
\end{longtable}

\hypertarget{hit-or-miss-monte-carlo}{%
\subsection{Hit-or-Miss Monte Carlo}\label{hit-or-miss-monte-carlo}}

Hit-or-Miss approach is another way to evaluate integrals.

\BeginKnitrBlock{example}[Estimation of $\pi$]
\protect\hypertarget{exm:estpi}{}{\label{exm:estpi} \iffalse (Estimation of \(\pi\)) \fi{} }Consider a circle in \(\mathbb{R}\) coordinate.

\[x^2 + y^2 = 1\]

Since \(y = \sqrt{1 - x^2}\),

\begin{equation}
  \int_0^1 \sqrt{1 - t^2} dt = \frac{\pi}{4}
  \label{eq:mcpi}
\end{equation}
\EndKnitrBlock{example}

By estimating Equation \eqref{eq:mcpi}, we can estimate \(\pi\), i.e.

\[\pi = 4 \int_0^1 \sqrt{1 - t^2} dt\]

Simple MC integration can also be used.

\begin{equation*}
  \begin{split}
    \int_0^1 \sqrt{1 - t^2} dt & = \int_0^1 \sqrt{1 - t^2} I_{(0,1)}(t) dt \\
    & \approx \frac{1}{N} \sum_{i = 1}^N \sqrt{1 - U_i^2}
  \end{split}
\end{equation*}

\begin{Shaded}
\begin{Highlighting}[]
\NormalTok{circ <-}\StringTok{ }\ControlFlowTok{function}\NormalTok{(x) \{}
  \DecValTok{4} \OperatorTok{*}\StringTok{ }\KeywordTok{sqrt}\NormalTok{(}\DecValTok{1} \OperatorTok{-}\StringTok{ }\NormalTok{x}\OperatorTok{^}\DecValTok{2}\NormalTok{)}
\NormalTok{\}}
\end{Highlighting}
\end{Shaded}

\begin{Shaded}
\begin{Highlighting}[]
\KeywordTok{tibble}\NormalTok{(}\DataTypeTok{x =} \KeywordTok{runif}\NormalTok{(}\DecValTok{10000}\NormalTok{)) }\OperatorTok\StringTok{ }
\StringTok{  }\KeywordTok{summarise}\NormalTok{(}\DataTypeTok{mc_pi =} \KeywordTok{mean}\NormalTok{(}\KeywordTok{circ}\NormalTok{(x)))}
\CommentTok{#> # A tibble: 1 x 1}
\CommentTok{#>   mc_pi}
\CommentTok{#>   <dbl>}
\CommentTok{#> 1  3.14}
\end{Highlighting}
\end{Shaded}

On the other way, hit-or-miss MC method applies geometric probability.

\begin{figure}[H]

{\centering \includegraphics[width=0.7\linewidth]{statistical-computing_files/figure-latex/hmmc-1} 

}

\caption{Hit-or-Miss}\label{fig:hmmc}
\end{figure}

See Figure \ref{fig:hmmc}. From each coordinate, generate

\begin{itemize}
\tightlist
\item
  \(X_i \stackrel{iid}{\sim}unif(0,1)\)
\item
  \(Y_i \stackrel{iid}{\sim}unif(0,1)\)
\end{itemize}

Then the proportion of \(Y_i \le \sqrt{1 - X_i^2}\) estimates \(\frac{\pi}{4}\).

\begin{Shaded}
\begin{Highlighting}[]
\KeywordTok{tibble}\NormalTok{(}\DataTypeTok{x =} \KeywordTok{runif}\NormalTok{(}\DecValTok{10000}\NormalTok{), }\DataTypeTok{y =} \KeywordTok{runif}\NormalTok{(}\DecValTok{10000}\NormalTok{)) }\OperatorTok\StringTok{ }
\StringTok{  }\KeywordTok{summarise}\NormalTok{(}\DataTypeTok{hitormiss =} \KeywordTok{mean}\NormalTok{(y }\OperatorTok{<=}\StringTok{ }\KeywordTok{sqrt}\NormalTok{(}\DecValTok{1} \OperatorTok{-}\StringTok{ }\NormalTok{x}\OperatorTok{^}\DecValTok{2}\NormalTok{)) }\OperatorTok{*}\StringTok{ }\DecValTok{4}\NormalTok{)}
\CommentTok{#> # A tibble: 1 x 1}
\CommentTok{#>   hitormiss}
\CommentTok{#>       <dbl>}
\CommentTok{#> 1      3.15}
\end{Highlighting}
\end{Shaded}

\hypertarget{variance-and-efficiency}{%
\section{Variance and Efficiency}\label{variance-and-efficiency}}

We have seen two apporoaches doing the same task. Now we want to \emph{evaluate them}. Denote that simple Monte Carlo integration \ref{thm:mcint} is estimating the \emph{expected value of some random variable}. Proportion, which approximates probability is expected value of identity function.

The common statistic that can evaluate estimators expected value might be their variances.

\hypertarget{variance}{%
\subsection{Variance}\label{variance}}

Note that variance of sample mean is \(Var(\overline{g(X)}) = \frac{Var(g(X))}{N}\). This property is one of estimating variance of \(\hat\theta\).

\begin{equation}
  \widehat{Var}(\hat\theta) = \frac{1}{N}\bigg( \frac{1}{N} \sum_{i = 1}^N (g(X_i) - \overline{g(X_i)}) \bigg) = \frac{1}{N^2} \sum_{i = 1}^N (g(X_i) - \overline{g(X_i)})
  \label{eq:mcsamvar}
\end{equation}

For example,

\begin{Shaded}
\begin{Highlighting}[]
\KeywordTok{tibble}\NormalTok{(}\DataTypeTok{x =} \KeywordTok{runif}\NormalTok{(}\DecValTok{10000}\NormalTok{)) }\OperatorTok\StringTok{ }
\StringTok{  }\KeywordTok{summarise}\NormalTok{(}\DataTypeTok{mc_pi =} \KeywordTok{var}\NormalTok{(}\KeywordTok{circ}\NormalTok{(x)) }\OperatorTok{/}\StringTok{ }\DecValTok{10000}\NormalTok{)}
\CommentTok{#> # A tibble: 1 x 1}
\CommentTok{#>       mc_pi}
\CommentTok{#>       <dbl>}
\CommentTok{#> 1 0.0000795}
\end{Highlighting}
\end{Shaded}

However, this \emph{variance of sample mean} is used in situation when we are in sample limitation situation. We do not have to stick to this. Now, Generating samples as many as we want is possible. So we try another approach: \emph{parametric bootstrap}.

\begin{figure}[H]

{\centering \includegraphics[width=0.7\linewidth]{images/mcint} 

}

\caption{Empircal distribution of $\hat\theta$}\label{fig:mcintvar}
\end{figure}

See Figure \ref{fig:mcintvar}. If we estimate \(E\Big[g(U \sim unif(a, b))\Big]\), we can get \(\theta\). Generate \(M\) samples \(\{ U_1^{(j)}, \ldots, U_N^{(j)} \}, j = 1, \ldots M\) from this \(U \sim unif(a, b)\). In each sample, calculate MC estimates \(\hat\theta^{(j)}\). Now we have \(M\) MC estimates \(\hat\theta\). This gives empirical distribution of \(\hat\theta\). By \emph{drawing a histogram}, we can see the outline.

\begin{algorithm}[H] \label{alg:algmcint}
  \SetAlgoLined
  \SetKwInOut{Input}{input}
  \SetKwInOut{Output}{output}
  \Input{$\theta = \int_a^b g(x) dx$}
  \For{$m \leftarrow 1$ \KwTo $M$}{
    Generate $U_1^{(m)}, \ldots, U_N^{(m)} \stackrel{iid}{\sim}unif(a, b)$ \;
    Compute $\hat\theta^{(j)} = \frac{(b - a)}{N} \sum g(U_i^{(j)})$\;
  }
  $\bar{\hat\theta} = \frac{1}{M} \sum \hat\theta^{(j)}$\;
  $\widehat{Var}(\hat\theta) = \frac{1}{M - 1} \sum (\hat\theta^{(j)} - \bar{\hat\theta})^2$\;
  \Output{$\widehat{Var}(\hat\theta)$}
  \caption{Variance of $\hat\theta$}
\end{algorithm}

Since we have to generate large size of data, \texttt{data.table} package will be used.

\begin{Shaded}
\begin{Highlighting}[]
\KeywordTok{library}\NormalTok{(data.table)}
\end{Highlighting}
\end{Shaded}

Group operation can be used. Additional column (\texttt{sam}) would indicate group, and for each group MC operation would be processed. The following is the function generating \texttt{data.table} before group operation.

\begin{Shaded}
\begin{Highlighting}[]
\NormalTok{mc_data <-}\StringTok{ }\ControlFlowTok{function}\NormalTok{(rand, }\DataTypeTok{N =} \DecValTok{10000}\NormalTok{, }\DataTypeTok{M =} \DecValTok{1000}\NormalTok{, }\DataTypeTok{char =} \StringTok{"s"}\NormalTok{, ...) \{}
  \KeywordTok{data.table}\NormalTok{(}
    \DataTypeTok{u =} \KeywordTok{rand}\NormalTok{(}\DataTypeTok{n =}\NormalTok{ N }\OperatorTok{*}\StringTok{ }\NormalTok{M, ...),}
    \DataTypeTok{sam =} \KeywordTok{gl}\NormalTok{(M, N, }\DataTypeTok{labels =} \KeywordTok{paste0}\NormalTok{(}\StringTok{"s"}\NormalTok{, }\DecValTok{1}\OperatorTok{:}\NormalTok{M))}
\NormalTok{  )}
\NormalTok{\}}
\end{Highlighting}
\end{Shaded}

\begin{Shaded}
\begin{Highlighting}[]
\NormalTok{pi_mc <-}
\StringTok{  }\KeywordTok{mc_data}\NormalTok{(runif)[,}
\NormalTok{                 .(}\DataTypeTok{mc_pi =} \KeywordTok{mean}\NormalTok{(}\KeywordTok{circ}\NormalTok{(u))),}
\NormalTok{                 keyby =}\StringTok{ }\NormalTok{sam]}
\end{Highlighting}
\end{Shaded}

\begin{Shaded}
\begin{Highlighting}[]
\NormalTok{pi_mc }\OperatorTok\StringTok{ }
\StringTok{  }\KeywordTok{ggplot}\NormalTok{(}\KeywordTok{aes}\NormalTok{(}\DataTypeTok{x =}\NormalTok{ mc_pi)) }\OperatorTok{+}
\StringTok{  }\KeywordTok{geom_histogram}\NormalTok{(}\DataTypeTok{bins =} \DecValTok{30}\NormalTok{, }\DataTypeTok{col =} \KeywordTok{gg_hcl}\NormalTok{(}\DecValTok{1}\NormalTok{), }\DataTypeTok{alpha =} \FloatTok{.7}\NormalTok{) }\OperatorTok{+}
\StringTok{  }\KeywordTok{xlab}\NormalTok{(}\KeywordTok{expression}\NormalTok{(pi)) }\OperatorTok{+}
\StringTok{  }\KeywordTok{geom_vline}\NormalTok{(}\DataTypeTok{xintercept =}\NormalTok{ pi, }\DataTypeTok{col =} \KeywordTok{gg_hcl}\NormalTok{(}\DecValTok{2}\NormalTok{)[}\DecValTok{2}\NormalTok{])}
\end{Highlighting}
\end{Shaded}

\begin{figure}[H]

{\centering \includegraphics[width=0.7\linewidth]{statistical-computing_files/figure-latex/smchis-1} 

}

\caption{Empirical distribution of $\hat\pi$ by simple MC}\label{fig:smchis}
\end{figure}

As in Algorighm \(\ref{alg:algmcint}\), we can compute the variance as below.

\begin{Shaded}
\begin{Highlighting}[]
\NormalTok{(mc_var <-}
\StringTok{  }\NormalTok{pi_mc[,}
\NormalTok{        .(}\DataTypeTok{mc_variance =} \KeywordTok{var}\NormalTok{(mc_pi))])}
\CommentTok{#>    mc_variance}
\CommentTok{#> 1:    8.09e-05}
\end{Highlighting}
\end{Shaded}

On the other hand, we need to generate two sets of random numbers for hit-or-miss MC.

\begin{Shaded}
\begin{Highlighting}[]
\NormalTok{pi_hit <-}
\StringTok{  }\KeywordTok{mc_data}\NormalTok{(runif)[}
\NormalTok{    , u2 }\OperatorTok{:}\ErrorTok{=}\StringTok{ }\KeywordTok{runif}\NormalTok{(}\DecValTok{10000} \OperatorTok{*}\StringTok{ }\DecValTok{1000}\NormalTok{)}
\NormalTok{  ][,}
\NormalTok{    .(}\DataTypeTok{hitormiss =} \KeywordTok{mean}\NormalTok{(u2 }\OperatorTok{<=}\StringTok{ }\KeywordTok{sqrt}\NormalTok{(}\DecValTok{1} \OperatorTok{-}\StringTok{ }\NormalTok{u}\OperatorTok{^}\DecValTok{2}\NormalTok{)) }\OperatorTok{*}\StringTok{ }\DecValTok{4}\NormalTok{),}
\NormalTok{    keyby =}\StringTok{ }\NormalTok{sam]}
\end{Highlighting}
\end{Shaded}

\begin{Shaded}
\begin{Highlighting}[]
\NormalTok{pi_mc[pi_hit] }\OperatorTok\StringTok{ }
\StringTok{  }\KeywordTok{melt}\NormalTok{(}\DataTypeTok{id.vars =} \StringTok{"sam"}\NormalTok{, }\DataTypeTok{variable.name =} \StringTok{"hat"}\NormalTok{) }\OperatorTok\StringTok{ }
\StringTok{  }\KeywordTok{ggplot}\NormalTok{(}\KeywordTok{aes}\NormalTok{(}\DataTypeTok{x =}\NormalTok{ value, }\DataTypeTok{fill =}\NormalTok{ hat)) }\OperatorTok{+}
\StringTok{  }\KeywordTok{geom_histogram}\NormalTok{(}\DataTypeTok{bins =} \DecValTok{30}\NormalTok{, }\DataTypeTok{alpha =} \FloatTok{.5}\NormalTok{, }\DataTypeTok{position =} \StringTok{"identity"}\NormalTok{) }\OperatorTok{+}
\StringTok{  }\KeywordTok{xlab}\NormalTok{(}\KeywordTok{expression}\NormalTok{(pi)) }\OperatorTok{+}
\StringTok{  }\KeywordTok{geom_vline}\NormalTok{(}\DataTypeTok{xintercept =}\NormalTok{ pi, }\DataTypeTok{col =} \KeywordTok{I}\NormalTok{(}\StringTok{"red"}\NormalTok{)) }\OperatorTok{+}
\StringTok{  }\KeywordTok{scale_fill_discrete}\NormalTok{(}
    \DataTypeTok{name =} \StringTok{"MC"}\NormalTok{,}
    \DataTypeTok{labels =} \KeywordTok{c}\NormalTok{(}\StringTok{"Simple"}\NormalTok{, }\StringTok{"Hit-or-Miss"}\NormalTok{)}
\NormalTok{  )}
\end{Highlighting}
\end{Shaded}

\begin{figure}[H]

{\centering \includegraphics[width=0.7\linewidth]{statistical-computing_files/figure-latex/simhit-1} 

}

\caption{Simple MC and Hit-or-Miss MC}\label{fig:simhit}
\end{figure}

\begin{Shaded}
\begin{Highlighting}[]
\NormalTok{(hit_var <-}
\StringTok{  }\NormalTok{pi_hit[,}
\NormalTok{         .(}\DataTypeTok{hit_variance =} \KeywordTok{var}\NormalTok{(hitormiss))])}
\CommentTok{#>    hit_variance}
\CommentTok{#> 1:     0.000258}
\end{Highlighting}
\end{Shaded}

\hypertarget{efficiency-1}{%
\subsection{Efficiency}\label{efficiency-1}}

See Figure \ref{fig:simhit}. It is obvious that Hit-or-Miss estimate produces larger variance than simple MC.

\BeginKnitrBlock{definition}[Efficiency]
\protect\hypertarget{def:eff}{}{\label{def:eff} \iffalse (Efficiency) \fi{} }Let \(\hat\theta_1\) and \(\hat\theta_2\) be two estimators for \(\theta\). Then \(\hat\theta_1\) is more efficient than \(\hat\theta_2\) if

\[\frac{Var(\hat\theta_1)}{Var(\hat\theta_2)} < 1\]
\EndKnitrBlock{definition}

In other words, if \(\hat\theta_1\) has smaller variance than \(\hat\theta_2\), then \(\hat\theta_1\) is said to be efficient, which is preferable.

\begin{longtable}{r|r|l}
\caption{\label{tab:unnamed-chunk-46}Simple MC versus Hit-or-Miss}\\
\hline
SimpleMC & Hit-or-Miss & SimpleMCefficiency\\
\hline
0 & 0 & TRUE\\
\hline
\end{longtable}

\hypertarget{variance-reduction}{%
\section{Variance Reduction}\label{variance-reduction}}

Consider Equation \eqref{eq:mcsamvar} based on \(Var(\hat\theta) = \frac{\sigma^2}{N}\). This variance can always reduced by adding \(N\). But we want to reduce variance less computationally.

\hypertarget{antithetic-variables}{%
\subsection{Antithetic Variables}\label{antithetic-variables}}

Consider correlated random variables \(U_1\) and \(U_2\). Then we have

\[Var\bigg( \frac{U_1 + U_2}{2} \bigg) = \frac{1}{4}\Big( Var(U_1) +  Var(U_2) + 2Cov(U_1, U_2)\Big)\]

See the last term \(Cov(U_1, U_2)\). If we generate \(U_{i1}\) and \(U_{i2}\) negatively correlated, we can get reduced variance than previous i.i.d. sample

\[Var\bigg( \frac{U_1 + U_2}{2} \bigg) = \frac{1}{4}\Big( Var(U_1) +  Var(U_2)\Big)\]

\BeginKnitrBlock{lemma}
\protect\hypertarget{lem:antiunif}{}{\label{lem:antiunif} }\(U\) and \(1 - U\) are identically distributed, but \emph{negatively correlated}.

\begin{enumerate}
  \item $U \sim unif(0,1) \Leftrightarrow 1 - U \sim unif(0,1)$
  \item $Corr(U, 1 - U) = -1$
\end{enumerate}
\EndKnitrBlock{lemma}

This is well-known property of uniform distribution. Instead of generating \(N\) uniform numbers, try \(\frac{N}{2}\) \(U_i\) and make corresponding \(\frac{N}{2}\) \(1 - U_i\). This sequence becomes negatively correlated, so we can reduce the variance as mentioned.

When can we replace previous numbers with these \emph{antithetic variables}? We usually plug-in the numbers in some function \(h\) to get Monte carlo integration. The thing is, our target is \(h\), not \(U\). \(h(U)\) and \(h(1 - U)\) should \emph{still be negatively correlated}. Hence, \(h\) should be \emph{monotonic function}.

\BeginKnitrBlock{corollary}
\protect\hypertarget{cor:antifun}{}{\label{cor:antifun} }If \(g = g(X_1, \ldots, X_n)\) is monotone, then

\[Y = g(F_X^{-1}(U_1), \ldots, F_X^{-1}(U_n))\]

and

\[Y^{\prime} = g(F_X^{-1}(1 - U_1), \ldots, F_X^{-1}(1 - U_n))\]

are negatively correlated.
\EndKnitrBlock{corollary}

\begin{algorithm}[H] \label{alg:alganti}
  \SetAlgoLined
  \SetKwInOut{Input}{input}
  \SetKwInOut{Output}{output}
  \Input{$h: monotonic$}
  \For{$m \leftarrow 1$ \KwTo $M$}{
    Generate $U_{1,1}^{(m)}, \ldots, U_{\frac{N}{2},1}^{(m)} \stackrel{iid}{\sim}unif(0, 1)$\;
    Set $U_{i,2}^{(m)} := 1 - U_{i,1}^{(m)} \stackrel{iid}{\sim}unif(0, 1)$\;
    $\{U_{i}^{(m)}\}_1^N = \{ U_{1,1}^{(m)}, \ldots, U_{{\frac{N}{2}},2}^{(m)} \}$\;
    $\hat\theta^{(j)} = \frac{1}{N} \sum h(U_i^{(j)})$\;
  }
  $\bar{\hat\theta} = \frac{1}{M} \sum \hat\theta^{(j)}$\;
  $\widehat{Var}(\hat\theta) = \frac{1}{M - 1} \sum (\hat\theta^{(j)} - \bar{\hat\theta})^2$\;
  \Output{$\widehat{Var}(\hat\theta)$}
  \caption{Variance of $\hat\theta$ using antithetic variables}
\end{algorithm}

Check again Example \ref{exm:mcex}. We have try to calculate

\[\Phi(x) = \int_{-\infty}^x \frac{1}{\sqrt{2\pi}}\exp\bigg(-\frac{t^2}{2}\bigg)dt\]

using simple monte carlo. To make the support \((0, 1)\), let \(y = \frac{t}{x}\) be a change of variable. Then

\begin{equation*}
  \begin{split}
    \int_0^x \exp\bigg(-\frac{t^2}{2}\bigg) dt & = \int_0^1 x\exp\bigg(-\frac{(xy)^2}{2}\bigg) dy \\
    & \approx \frac{1}{N} \sum_{i = 1}^N x\exp\bigg(-\frac{(xU_i)^2}{2}\bigg)
  \end{split}
\end{equation*}

\begin{Shaded}
\begin{Highlighting}[]
\NormalTok{phiunif <-}\StringTok{ }\ControlFlowTok{function}\NormalTok{(x, y) \{}
\NormalTok{  yi <-}\StringTok{ }\KeywordTok{abs}\NormalTok{(y)}
\NormalTok{  theta <-}\StringTok{ }\KeywordTok{mean}\NormalTok{(yi }\OperatorTok{*}\StringTok{ }\KeywordTok{exp}\NormalTok{(}\OperatorTok{-}\NormalTok{(yi }\OperatorTok{*}\StringTok{ }\NormalTok{x)}\OperatorTok{^}\DecValTok{2} \OperatorTok{/}\StringTok{ }\DecValTok{2}\NormalTok{)) }\OperatorTok{/}\StringTok{ }\KeywordTok{sqrt}\NormalTok{(}\DecValTok{2} \OperatorTok{*}\StringTok{ }\NormalTok{pi) }\OperatorTok{+}\StringTok{ }\FloatTok{.5}
  \KeywordTok{ifelse}\NormalTok{(y }\OperatorTok{>=}\StringTok{ }\DecValTok{0}\NormalTok{, theta, }\DecValTok{1} \OperatorTok{-}\StringTok{ }\NormalTok{theta)}
\NormalTok{\}}
\end{Highlighting}
\end{Shaded}

Consider \(\Phi(2)\).

\begin{Shaded}
\begin{Highlighting}[]
\NormalTok{phi2 <-}
\StringTok{  }\KeywordTok{mc_data}\NormalTok{(runif)[,}
\NormalTok{                 .(}\DataTypeTok{p2 =} \KeywordTok{phiunif}\NormalTok{(u, }\DataTypeTok{y =} \DecValTok{2}\NormalTok{)),}
\NormalTok{                 keyby =}\StringTok{ }\NormalTok{sam]}
\end{Highlighting}
\end{Shaded}

Now apply antithetic variables.

\begin{Shaded}
\begin{Highlighting}[]
\NormalTok{phi2_anti <-}
\StringTok{  }\KeywordTok{mc_data}\NormalTok{(runif, }\DataTypeTok{N =} \DecValTok{10000} \OperatorTok{/}\StringTok{ }\DecValTok{2}\NormalTok{)[,}
\NormalTok{                                u2 }\OperatorTok{:}\ErrorTok{=}\StringTok{ }\DecValTok{1} \OperatorTok{-}\StringTok{ }\NormalTok{u] }\OperatorTok\StringTok{ }
\StringTok{  }\KeywordTok{melt}\NormalTok{(}\DataTypeTok{id.vars =} \StringTok{"sam"}\NormalTok{, }\DataTypeTok{value.name =} \StringTok{"U"}\NormalTok{) }\OperatorTok\StringTok{ }
\StringTok{  }\NormalTok{.[,}
\NormalTok{    .(}\DataTypeTok{anti_p2 =} \KeywordTok{phiunif}\NormalTok{(U, }\DataTypeTok{y =} \DecValTok{2}\NormalTok{)),}
\NormalTok{    keyby =}\StringTok{ }\NormalTok{sam]}
\end{Highlighting}
\end{Shaded}

\begin{Shaded}
\begin{Highlighting}[]
\NormalTok{phi2[phi2_anti] }\OperatorTok\StringTok{ }
\StringTok{  }\KeywordTok{melt}\NormalTok{(}\DataTypeTok{id.vars =} \StringTok{"sam"}\NormalTok{, }\DataTypeTok{variable.name =} \StringTok{"hat"}\NormalTok{) }\OperatorTok\StringTok{ }
\StringTok{  }\KeywordTok{ggplot}\NormalTok{(}\KeywordTok{aes}\NormalTok{(}\DataTypeTok{x =}\NormalTok{ value, }\DataTypeTok{fill =}\NormalTok{ hat)) }\OperatorTok{+}
\StringTok{  }\KeywordTok{geom_histogram}\NormalTok{(}\DataTypeTok{bins =} \DecValTok{30}\NormalTok{, }\DataTypeTok{alpha =} \FloatTok{.5}\NormalTok{, }\DataTypeTok{position =} \StringTok{"identity"}\NormalTok{) }\OperatorTok{+}
\StringTok{  }\KeywordTok{xlab}\NormalTok{(}\KeywordTok{expression}\NormalTok{(pi)) }\OperatorTok{+}
\StringTok{  }\KeywordTok{geom_vline}\NormalTok{(}\DataTypeTok{xintercept =} \KeywordTok{pnorm}\NormalTok{(}\DecValTok{2}\NormalTok{), }\DataTypeTok{col =} \KeywordTok{I}\NormalTok{(}\StringTok{"red"}\NormalTok{)) }\OperatorTok{+}
\StringTok{  }\KeywordTok{scale_fill_discrete}\NormalTok{(}
    \DataTypeTok{name =} \StringTok{"MC"}\NormalTok{,}
    \DataTypeTok{labels =} \KeywordTok{c}\NormalTok{(}\StringTok{"Simple"}\NormalTok{, }\StringTok{"Antithetic"}\NormalTok{)}
\NormalTok{  )}
\end{Highlighting}
\end{Shaded}

\begin{figure}[H]

{\centering \includegraphics[width=0.7\linewidth]{statistical-computing_files/figure-latex/svsanti-1} 

}

\caption{Use of antithetic variables}\label{fig:svsanti}
\end{figure}

Obviously, variance has been reduced.

\begin{Shaded}
\begin{Highlighting}[]
\NormalTok{phi2[phi2_anti] }\OperatorTok\StringTok{ }
\StringTok{  }\KeywordTok{melt}\NormalTok{(}\DataTypeTok{id.vars =} \StringTok{"sam"}\NormalTok{, }\DataTypeTok{variable.name =} \StringTok{"hat"}\NormalTok{) }\OperatorTok\StringTok{ }
\StringTok{  }\NormalTok{.[,}
\NormalTok{    .(}\DataTypeTok{variance =} \KeywordTok{var}\NormalTok{(value)),}
\NormalTok{    by =}\StringTok{ }\NormalTok{hat]}
\CommentTok{#>        hat variance}
\CommentTok{#> 1:      p2  5.2e-06}
\CommentTok{#> 2: anti_p2  1.5e-08}
\end{Highlighting}
\end{Shaded}

\hypertarget{control-variates}{%
\subsection{Control Variates}\label{control-variates}}

Recall that we are trying to estimate \(\theta = EX\) here in MC integration. Consider other output random variable. Suppose that \(\mu_Y \equiv E(Y)\) is known. Then

\[X + c(Y - \mu_Y)\]

is an unbiased estimator for \(\theta\) for any \(c \in \mathbb{R}\).

\hypertarget{importance-sampling}{%
\section{Importance Sampling}\label{importance-sampling}}

\hypertarget{monte-carlo-methods-in-inference}{%
\chapter{Monte Carlo Methods in Inference}\label{monte-carlo-methods-in-inference}}

\hypertarget{parametric-bootstrap}{%
\section{Parametric Bootstrap}\label{parametric-bootstrap}}

In this setting, we know distribution of \(X\). We can freely generate from this distribution.

\begin{figure}[H]

{\centering \includegraphics[width=0.7\linewidth]{images/mcboot} 

}

\caption{Parametric bootstrap}\label{fig:paramboot}
\end{figure}

See Figure \ref{fig:paramboot}. From the ``true'' distribution, we can generate multiple samples. From each sample estimator can be computed. Then we can check these multiple estimates. Multiple estimates are close to motivation of estimator, so it helps exploring statistical inference with simple steps.

\begin{Shaded}
\begin{Highlighting}[]
\NormalTok{mc_data <-}\StringTok{ }\ControlFlowTok{function}\NormalTok{(rand, }\DataTypeTok{N =} \DecValTok{10000}\NormalTok{, }\DataTypeTok{M =} \DecValTok{1000}\NormalTok{, }\DataTypeTok{char =} \StringTok{"s"}\NormalTok{, ...) \{}
  \KeywordTok{data.table}\NormalTok{(}
    \DataTypeTok{x =} \KeywordTok{rand}\NormalTok{(}\DataTypeTok{n =}\NormalTok{ N }\OperatorTok{*}\StringTok{ }\NormalTok{M, ...),}
    \DataTypeTok{sam =} \KeywordTok{gl}\NormalTok{(M, N, }\DataTypeTok{labels =} \KeywordTok{paste0}\NormalTok{(}\StringTok{"s"}\NormalTok{, }\DecValTok{1}\OperatorTok{:}\NormalTok{M))}
\NormalTok{  )}
\NormalTok{\}}
\end{Highlighting}
\end{Shaded}

\hypertarget{monte-carlo-methods-for-estimation}{%
\section{Monte Carlo Methods for Estimation}\label{monte-carlo-methods-for-estimation}}

\BeginKnitrBlock{example}[Any quantity of interest]
\protect\hypertarget{exm:quanint}{}{\label{exm:quanint} \iffalse (Any quantity of interest) \fi{} }Suppose that \(X_1, X_2 \stackrel{iid}{\sim}N(0, 1)\). We want to estimate

\[\theta = E\lvert X_1 - X_2 \rvert\]
\EndKnitrBlock{example}

\hypertarget{empirical-distribution}{%
\subsection{Empirical distribution}\label{empirical-distribution}}

\begin{algorithm}[H] \label{alg:algx1x2}
  \SetAlgoLined
  \SetKwInOut{Input}{input}
  \SetKwInOut{Output}{output}
  \Input{distribution $f$}
  \For{$m \leftarrow 1$ \KwTo $M$}{
    Generate $(X_1^{(m)}, X_2^{(m)}) \stackrel{iid}{\sim}N(0, 1)$\;
    Compute $\hat\theta^{(m)} = \lvert X_1^{(m)} - X_2^{(m)} \rvert$\;
  }
  Draw a histogram\;
  \Output{$\bar{\hat\theta} = \frac{1}{M} \sum\limits_{m = 1}^M\hat\theta_m^{(m)}, \{ \hat\theta^{(1)}, \ldots, \hat\theta^{(M)} \}$}
  \caption{Empirical distribution of $\hat\theta$}
\end{algorithm}

\begin{Shaded}
\begin{Highlighting}[]
\NormalTok{basicmc <-}
\StringTok{  }\KeywordTok{mc_data}\NormalTok{(rnorm, }\DataTypeTok{N =} \DecValTok{2}\NormalTok{)[,}
\NormalTok{                        xname }\OperatorTok{:}\ErrorTok{=}\StringTok{ }\KeywordTok{gl}\NormalTok{(}\DecValTok{2}\NormalTok{, }\DecValTok{1}\NormalTok{, }\DataTypeTok{length =} \DecValTok{2000}\NormalTok{, }\DataTypeTok{labels =} \KeywordTok{c}\NormalTok{(}\StringTok{"x1"}\NormalTok{, }\StringTok{"x2"}\NormalTok{))] }\OperatorTok\StringTok{ }
\StringTok{  }\KeywordTok{dcast}\NormalTok{(sam }\OperatorTok{~}\StringTok{ }\NormalTok{xname, }\DataTypeTok{value.var =} \StringTok{"x"}\NormalTok{) }\OperatorTok\StringTok{ }
\StringTok{  }\NormalTok{.[,}
\NormalTok{    .(}\DataTypeTok{that =} \KeywordTok{mean}\NormalTok{(}\KeywordTok{abs}\NormalTok{(x1 }\OperatorTok{-}\StringTok{ }\NormalTok{x2))),}
\NormalTok{    by =}\StringTok{ }\NormalTok{sam]}
\end{Highlighting}
\end{Shaded}

\begin{Shaded}
\begin{Highlighting}[]
\NormalTok{basicmc[,}
\NormalTok{        .(}\DataTypeTok{est =} \KeywordTok{mean}\NormalTok{(that))]}
\CommentTok{#>    est}
\CommentTok{#> 1: 1.1}
\end{Highlighting}
\end{Shaded}

\begin{Shaded}
\begin{Highlighting}[]
\NormalTok{basicmc }\OperatorTok\StringTok{ }
\StringTok{  }\KeywordTok{ggplot}\NormalTok{(}\KeywordTok{aes}\NormalTok{(}\DataTypeTok{x =}\NormalTok{ that)) }\OperatorTok{+}
\StringTok{  }\KeywordTok{geom_histogram}\NormalTok{(}\DataTypeTok{bins =} \DecValTok{30}\NormalTok{, }\DataTypeTok{col =} \KeywordTok{gg_hcl}\NormalTok{(}\DecValTok{1}\NormalTok{), }\DataTypeTok{alpha =} \FloatTok{.7}\NormalTok{) }\OperatorTok{+}
\StringTok{  }\KeywordTok{xlab}\NormalTok{(}\KeywordTok{expression}\NormalTok{(theta))}
\end{Highlighting}
\end{Shaded}

\begin{figure}[H]

{\centering \includegraphics[width=0.7\linewidth]{statistical-computing_files/figure-latex/absx12-1} 

}

\caption{Empirical distribution of $\hat\theta$ for $\lvert X_1 - X_2 \rvert$}\label{fig:absx12}
\end{figure}

\hypertarget{standard-error}{%
\subsection{Standard error}\label{standard-error}}

In Algorithm \(\ref{alg:algx1x2}\), we can get standard error by just calculating standard deviation of

\[\{ \hat\theta^{(1)}, \ldots, \hat\theta^{(M)} \}\]

\begin{algorithm}[H] \label{alg:algmcse}
  \SetAlgoLined
  \SetKwInOut{Input}{input}
  \SetKwInOut{Output}{output}
  \Input{distribution $f$}
  \For{$m \leftarrow 1$ \KwTo $M$}{
    Generate $(X_1^{(m)}, X_2^{(m)}) \stackrel{iid}{\sim}N(0, 1)$\;
    Compute $\hat\theta^{(m)} = \lvert X_1^{(m)} - X_2^{(m)} \rvert$\;
  }
  $\bar{\hat\theta} = \frac{1}{M} \sum\limits_{m = 1}^M\hat\theta_m^{(m)}$\;
  $\widehat{SE}(\hat\theta) = \sqrt{\frac{1}{M - 1}\sum\limits_{m = 1}^M(\hat\theta^{(m)} - \bar{\hat\theta})}$\;
  \Output{$\widehat{SE}(\hat\theta)$}
  \caption{Standard error of $\hat\theta$}
\end{algorithm}

\begin{Shaded}
\begin{Highlighting}[]
\NormalTok{basicmc[,}
\NormalTok{        .(}\DataTypeTok{se =} \KeywordTok{sd}\NormalTok{(that))]}
\CommentTok{#>       se}
\CommentTok{#> 1: 0.844}
\end{Highlighting}
\end{Shaded}

\hypertarget{mean-squared-error}{%
\subsection{Mean squared error}\label{mean-squared-error}}

\(MSE\) is used when comparing several estimators.

\BeginKnitrBlock{definition}[Mean squared error]
\protect\hypertarget{def:mse}{}{\label{def:mse} \iffalse (Mean squared error) \fi{} }\[MSE(\hat\theta) := E(\hat\theta - \theta)^2\]
\EndKnitrBlock{definition}

To know \(MSE\), however, we should compute expectation. Some of them might be complicated even though we know true distribution. As the last chapter, we can apply Monte carlo method.

\BeginKnitrBlock{example}[MSE of a trimmed mean]
\protect\hypertarget{exm:trim}{}{\label{exm:trim} \iffalse (MSE of a trimmed mean) \fi{} }Suppose that \(X_1, \ldots, X_n \stackrel{iid}{\sim}N(2, 1)\). Consider three estimators for \(\mu = 2\).

\begin{enumerate}
  \item mean $\overline{X}$
  \item median $\tilde{X}$
  \item $k$th trimmed mean $\overline{X}_{[-k]}$
\end{enumerate}
\EndKnitrBlock{example}

\begin{algorithm}[H] \label{alg:algmse}
  \SetAlgoLined
  \SetKwInOut{Input}{input}
  \SetKwInOut{Output}{output}
  \Input{distribution $f$}
  \For{$m \leftarrow 1$ \KwTo $M$}{
    Generate $(X_1^{(m)}, \ldots, X_N^{(m)}) \stackrel{iid}{\sim}N(2, 1)$\;
    Sort $(X_1^{(m)}, \ldots, X_N^{(m)})$ in increasing order, i.e. $(X_{(1)}^{(m)}, \ldots, X_{(N)}^{(m)})$\;
    Mean $\overline{X}^{(m)} = \frac{1}{N}\sum\limits_{i = 1}^N X_i^{(m)}$\;
    Median $\tilde{X}^{(m)} = \begin{cases} X_{\frac{N}{2} + 1}^{(m)} & N \:\text{odd} \\ \frac{X_{\frac{N}{2}}^{(m)} + X_{\frac{N}{2} + 1}^{(m)}}{2} & N \:\text{even} \end{cases}$\;
    $k$th trimmed mean $\overline{X}_{[-k]}^{(m)} = \frac{1}{N - 2k}\sum\limits_{i = k + 1}^{n - k}X_{(i)}^{(m)}$
  }
  $\widehat{MSE}(\overline{X}) = \frac{1}{M} \sum\limits_{m = 1}^M (\overline{X}^{(m)} - 2)^2$\;
  $\widehat{MSE}(\tilde{X}) = \frac{1}{M} \sum\limits_{m = 1}^M (\tilde{X}^{(m)} - 2)^2$\;
  $\widehat{MSE}(\overline{X}_{[-k]}) = \frac{1}{M} \sum\limits_{m = 1}^M (\overline{X}_{[-k]}^{(m)} - 2)^2$\;
  \Output{$\widehat{MSE}(\overline{X}), \widehat{MSE}(\tilde{X}), \:\text{and}\: \widehat{MSE}(\overline{X}_{[-k]})$}
  \caption{MSE of mean, median, and $k$th trimmed mean}
\end{algorithm}

\begin{Shaded}
\begin{Highlighting}[]
\NormalTok{trim <-}\StringTok{ }\ControlFlowTok{function}\NormalTok{(x, }\DataTypeTok{k =} \DecValTok{1}\NormalTok{) \{}
\NormalTok{  n <-}\StringTok{ }\KeywordTok{length}\NormalTok{(x)}
\NormalTok{  x <-}\StringTok{ }\KeywordTok{sort}\NormalTok{(x)}
  \KeywordTok{sum}\NormalTok{(x[(k }\OperatorTok{+}\StringTok{ }\DecValTok{1}\NormalTok{)}\OperatorTok{:}\NormalTok{(n }\OperatorTok{-}\StringTok{ }\NormalTok{k)]) }\OperatorTok{/}\StringTok{ }\NormalTok{(n }\OperatorTok{-}\StringTok{ }\DecValTok{2} \OperatorTok{*}\StringTok{ }\NormalTok{k)}
\NormalTok{\}}
\CommentTok{#--------------------------------------}
\NormalTok{mu_list <-}\StringTok{ }\ControlFlowTok{function}\NormalTok{(x, k) \{}
  \KeywordTok{list}\NormalTok{(}\DataTypeTok{mean =} \KeywordTok{mean}\NormalTok{(x), }\DataTypeTok{median =} \KeywordTok{median}\NormalTok{(x), }\DataTypeTok{trim =} \KeywordTok{trim}\NormalTok{(x, k))}
\NormalTok{\}}
\end{Highlighting}
\end{Shaded}

Try \(k = 1\).

\begin{Shaded}
\begin{Highlighting}[]
\NormalTok{(trim_mc <-}
\StringTok{  }\KeywordTok{mc_data}\NormalTok{(rnorm, }\DataTypeTok{mean =} \DecValTok{2}\NormalTok{, }\DataTypeTok{sd =} \DecValTok{1}\NormalTok{)[,}
                                   \KeywordTok{unlist}\NormalTok{(}\KeywordTok{lapply}\NormalTok{(.SD, mu_list, }\DataTypeTok{k =} \DecValTok{1}\NormalTok{)) }\OperatorTok\StringTok{ }\NormalTok{as.list,}
                                   \DataTypeTok{by =}\NormalTok{ sam])}
\CommentTok{#>         sam x.mean x.median x.trim}
\CommentTok{#>    1:    s1   2.02     2.02   2.02}
\CommentTok{#>    2:    s2   2.00     2.00   2.00}
\CommentTok{#>    3:    s3   2.00     2.01   2.00}
\CommentTok{#>    4:    s4   1.99     1.98   1.99}
\CommentTok{#>    5:    s5   2.00     1.99   2.00}
\CommentTok{#>   ---                             }
\CommentTok{#>  996:  s996   2.02     2.02   2.02}
\CommentTok{#>  997:  s997   2.00     1.99   2.00}
\CommentTok{#>  998:  s998   1.99     1.99   1.99}
\CommentTok{#>  999:  s999   1.99     1.99   1.99}
\CommentTok{#> 1000: s1000   2.00     2.01   2.00}
\end{Highlighting}
\end{Shaded}

\begin{Shaded}
\begin{Highlighting}[]
\NormalTok{trim_mc }\OperatorTok\StringTok{ }
\StringTok{  }\KeywordTok{melt}\NormalTok{(}\DataTypeTok{id.vars =} \StringTok{"sam"}\NormalTok{, }\DataTypeTok{variable.name =} \StringTok{"hat"}\NormalTok{) }\OperatorTok\StringTok{ }
\StringTok{  }\KeywordTok{ggplot}\NormalTok{(}\KeywordTok{aes}\NormalTok{(}\DataTypeTok{x =}\NormalTok{ value, }\DataTypeTok{fill =}\NormalTok{ hat)) }\OperatorTok{+}
\StringTok{  }\KeywordTok{geom_histogram}\NormalTok{(}\DataTypeTok{bins =} \DecValTok{30}\NormalTok{, }\DataTypeTok{alpha =} \FloatTok{.3}\NormalTok{, }\DataTypeTok{position =} \StringTok{"identity"}\NormalTok{) }\OperatorTok{+}
\StringTok{  }\KeywordTok{xlab}\NormalTok{(}\KeywordTok{expression}\NormalTok{(mu)) }\OperatorTok{+}
\StringTok{  }\KeywordTok{geom_vline}\NormalTok{(}\DataTypeTok{xintercept =} \DecValTok{2}\NormalTok{, }\DataTypeTok{col =} \KeywordTok{I}\NormalTok{(}\StringTok{"red"}\NormalTok{)) }\OperatorTok{+}
\StringTok{  }\KeywordTok{scale_fill_discrete}\NormalTok{(}
    \DataTypeTok{name =} \StringTok{"Estimates"}\NormalTok{,}
    \DataTypeTok{labels =} \KeywordTok{c}\NormalTok{(}\StringTok{"Mean"}\NormalTok{, }\StringTok{"Median"}\NormalTok{, }\StringTok{"Trimmed"}\NormalTok{)}
\NormalTok{  )}
\end{Highlighting}
\end{Shaded}

\begin{figure}[H]

{\centering \includegraphics[width=0.7\linewidth]{statistical-computing_files/figure-latex/meanemp-1} 

}

\caption{Empirical distribution of each estimator for $\mu = 2$}\label{fig:meanemp}
\end{figure}

Here, median shows the largest standard error.

\begin{Shaded}
\begin{Highlighting}[]
\NormalTok{trim_mc[,}
        \KeywordTok{lapply}\NormalTok{(.SD, sd),}
\NormalTok{        .SDcols =}\StringTok{ }\OperatorTok{-}\StringTok{"sam"}\NormalTok{]}
\CommentTok{#>     x.mean x.median  x.trim}
\CommentTok{#> 1: 0.00942   0.0123 0.00942}
\end{Highlighting}
\end{Shaded}

Now try various \(k\) for trimmed mean.

\begin{Shaded}
\begin{Highlighting}[]
\NormalTok{mse_list <-}\StringTok{ }\ControlFlowTok{function}\NormalTok{(x, k) \{}
  \KeywordTok{list}\NormalTok{(}\DataTypeTok{mse =} \KeywordTok{mean}\NormalTok{((x }\OperatorTok{-}\StringTok{ }\DecValTok{2}\NormalTok{)}\OperatorTok{^}\DecValTok{2}\NormalTok{), }\DataTypeTok{se =} \KeywordTok{sd}\NormalTok{(x))}
\NormalTok{\}}
\CommentTok{#-----------------------------------------}
\NormalTok{trim_mse <-}
\StringTok{  }\KeywordTok{mc_data}\NormalTok{(rnorm, }\DataTypeTok{mean =} \DecValTok{2}\NormalTok{, }\DataTypeTok{sd =} \DecValTok{1}\NormalTok{)[,}
                                   \KeywordTok{lapply}\NormalTok{(.SD, }\ControlFlowTok{function}\NormalTok{(x) \{}
                                     \KeywordTok{sapply}\NormalTok{(}\DecValTok{0}\OperatorTok{:}\DecValTok{9}\NormalTok{, }\ControlFlowTok{function}\NormalTok{(k) \{}
                                       \KeywordTok{trim}\NormalTok{(}\DataTypeTok{x =}\NormalTok{ x, }\DataTypeTok{k =}\NormalTok{ k)}
\NormalTok{                                     \})}
\NormalTok{                                   \}) }\OperatorTok\StringTok{ }
\StringTok{                                     }\KeywordTok{unlist}\NormalTok{() }\OperatorTok\StringTok{ }
\StringTok{                                     }\KeywordTok{as.list}\NormalTok{(),}
\NormalTok{                                   by =}\StringTok{ }\NormalTok{sam][,}
                                             \KeywordTok{lapply}\NormalTok{(.SD, mse_list) }\OperatorTok\StringTok{ }
\StringTok{                                               }\KeywordTok{unlist}\NormalTok{() }\OperatorTok\StringTok{ }
\StringTok{                                               }\KeywordTok{as.list}\NormalTok{(),}
\NormalTok{                                             .SDcols =}\StringTok{ }\OperatorTok{-}\StringTok{"sam"}\NormalTok{]}
\end{Highlighting}
\end{Shaded}

\begin{Shaded}
\begin{Highlighting}[]
\NormalTok{trim_mse }\OperatorTok\StringTok{ }
\StringTok{  }\KeywordTok{transpose}\NormalTok{() }\OperatorTok\StringTok{ }
\StringTok{  }\NormalTok{.[,}
    \StringTok{`}\DataTypeTok{:=}\StringTok{`}\NormalTok{(}
      \DataTypeTok{k =} \KeywordTok{rep}\NormalTok{(}\DecValTok{0}\OperatorTok{:}\DecValTok{9}\NormalTok{, }\DataTypeTok{each =} \DecValTok{2}\NormalTok{),}
      \DataTypeTok{hat =} \KeywordTok{gl}\NormalTok{(}\DecValTok{2}\NormalTok{, }\DataTypeTok{k =} \DecValTok{1}\NormalTok{, }\DataTypeTok{length =} \DecValTok{2} \OperatorTok{*}\StringTok{ }\DecValTok{10}\NormalTok{, }\DataTypeTok{labels =} \KeywordTok{c}\NormalTok{(}\StringTok{"mse"}\NormalTok{, }\StringTok{"se"}\NormalTok{))}
\NormalTok{    )] }\OperatorTok\StringTok{ }
\StringTok{  }\KeywordTok{dcast}\NormalTok{(k }\OperatorTok{~}\StringTok{ }\NormalTok{hat, }\DataTypeTok{value.var =} \StringTok{"V1"}\NormalTok{)}
\CommentTok{#>     k      mse      se}
\CommentTok{#>  1: 0 9.83e-05 0.00992}
\CommentTok{#>  2: 1 9.83e-05 0.00992}
\CommentTok{#>  3: 2 9.83e-05 0.00992}
\CommentTok{#>  4: 3 9.83e-05 0.00992}
\CommentTok{#>  5: 4 9.82e-05 0.00992}
\CommentTok{#>  6: 5 9.83e-05 0.00992}
\CommentTok{#>  7: 6 9.83e-05 0.00992}
\CommentTok{#>  8: 7 9.83e-05 0.00992}
\CommentTok{#>  9: 8 9.83e-05 0.00992}
\CommentTok{#> 10: 9 9.83e-05 0.00992}
\end{Highlighting}
\end{Shaded}

\hypertarget{confidence-interval}{%
\section{Confidence interval}\label{confidence-interval}}

Remember the meaning of 95\% confidence interval. \emph{If we have 100 samples and construct confidence interval in each sample, 95 intervals would include true parameter}. In this Monte Carlo setting, we know true population distribution, so we can generate multiple samples. Thus, we can reproduce this confidence interval situation.

\hypertarget{empirical-confidence-interval}. Best critical region leads to the shortest length of CI given \(\alpha\), so we are finding this one.

\begin{algorithm}[H] \label{alg:algempci}
  \SetAlgoLined
  \SetKwInOut{Input}{input}
  \SetKwInOut{Output}{output}
  \Input{distribution $f$}
  \For{$m \leftarrow 1$ \KwTo $M$}{
    Generate $X_1^{(m)}, \ldots, X_n^{(m)} \stackrel{iid}{\sim}f$\;
    Compute $\hat\theta^{(m)} = \hat\theta(\mathbf{\mathbf{X}^{(m)}})$\;
  }
  \eIf{Distribution of $\{ \hat\theta^{(m)} \}_1^M$ symmetric}{
    Sort $\{ \hat\theta^{(1)}, \ldots, \hat\theta^{(M)} \}$ in decreasing order, i.e. $\{ \hat\theta_{(1)}^{(1)}, \ldots, \hat\theta_{(M)}^{(M)} \}$\;
    Compute $LB= \frac{\alpha}{2} \:\text{sample quantile}$ and $UB= 1 - \frac{\alpha}{2} \:\text{sample quantile}$\;
  }{
    \ForEach{$lb < 0.05$ with $ub - lb = 1 - \alpha$}{
      Candidate interval $(lb, ub)$\;
      calculate length $l_i = ub - lb$\;
    }
    $(LB, UB)$: pick up the interval with the smallest length $l_i$\;
  }
  \Output{$(LB, UB)$}
  \caption{Empirical confidence interval by Monte Carlo method}
\end{algorithm}

\hypertarget{empirical-confidence-level}{%
\subsection{Empirical confidence level}\label{empirical-confidence-level}}

On the contrary, we can estiamte confidence level given confidence interval.

\BeginKnitrBlock{example}[Confidence interval for variance]
\protect\hypertarget{exm:civar}{}{\label{exm:civar} \iffalse (Confidence interval for variance) \fi{} }If \(X_1, \ldots, X_n \stackrel{iid}{\sim}N(\mu, \sigma^2)\), then

\[T = \frac{(n - 1)S^2}{\sigma^2} \sim \chi^2(n - 1)\]

Thus, \(100(1 - \alpha)\%\) confidence interval is given by

\[(0, \frac{(n -1)S^2}{\chi^2_{\alpha}(n - 1)})\]
\EndKnitrBlock{example}

For each MC sample, compute confidence interval. Just check if \emph{known true parameter} is in the interval. Its proportion becomes the confidence level. It is simpler that estimate confidence interval itself.

\begin{algorithm}[H] \label{alg:algcilev}
  \SetAlgoLined
  \SetKwInOut{Input}{input}
  \SetKwInOut{Output}{output}
  \Input{distribution $f$ with parameter $\theta$}
  \For{$m \leftarrow 1$ \KwTo $M$}{
    Generate $X_1^{(m)}, \ldots, X_n^{(m)} \stackrel{iid}{\sim}f$\;
    Compute the confidence interval $C_m$\;
    Compute $Y_j = I(\theta \in C_m)$, i.e. whether $\theta$ is in the CI\;
  }
  Empirical confidence level $\overline{Y} = \sum\limits_{m = 1}^M Y_m$\; \label{alg:cilevlast}
  \Output{$\overline{Y}$}
  \caption{Empirical confidence level by Monte Carlo method}
\end{algorithm}

Let \(\mu = 0\), \(\sigma = 2\), \(N = 20\), and let \(M = 1000\).

\begin{Shaded}
\begin{Highlighting}[]
\NormalTok{ci_var <-}\StringTok{ }\ControlFlowTok{function}\NormalTok{(x, variance, alpha) \{}
\NormalTok{  n <-}\StringTok{ }\KeywordTok{length}\NormalTok{(x)}
\NormalTok{  s2 <-}\StringTok{ }\KeywordTok{var}\NormalTok{(x)}
\NormalTok{  (n }\OperatorTok{-}\StringTok{ }\DecValTok{1}\NormalTok{) }\OperatorTok{*}\StringTok{ }\NormalTok{s2 }\OperatorTok{/}\StringTok{ }\KeywordTok{qchisq}\NormalTok{(alpha, }\DataTypeTok{df =}\NormalTok{ n }\OperatorTok{-}\StringTok{ }\DecValTok{1}\NormalTok{) }\OperatorTok{>}\StringTok{ }\NormalTok{variance}
\NormalTok{\}}
\CommentTok{#---------------------------}
\NormalTok{ci_lev <-}
\StringTok{  }\KeywordTok{mc_data}\NormalTok{(rnorm, }\DataTypeTok{N =} \DecValTok{20}\NormalTok{, }\DataTypeTok{M =} \DecValTok{1000}\NormalTok{, }\DataTypeTok{mean =} \DecValTok{0}\NormalTok{, }\DataTypeTok{sd =} \DecValTok{2}\NormalTok{)[,}
\NormalTok{                                                     .(}\DataTypeTok{hat =} \KeywordTok{mean}\NormalTok{(}\KeywordTok{ci_var}\NormalTok{(x, }\DataTypeTok{variance =} \DecValTok{4}\NormalTok{, }\DataTypeTok{alpha =} \FloatTok{.05}\NormalTok{))),}
\NormalTok{                                                     by =}\StringTok{ }\NormalTok{sam]}
\end{Highlighting}
\end{Shaded}

\begin{Shaded}
\begin{Highlighting}[]
\NormalTok{ci_lev[,}
\NormalTok{       .N,}
\NormalTok{       by =}\StringTok{ }\NormalTok{hat][,}
\NormalTok{                 proportion }\OperatorTok{:}\ErrorTok{=}\StringTok{ }\NormalTok{N }\OperatorTok{/}\StringTok{ }\KeywordTok{sum}\NormalTok{(N)] }\OperatorTok
\StringTok{  }\KeywordTok{ggplot}\NormalTok{(}\KeywordTok{aes}\NormalTok{(}\DataTypeTok{x =}\NormalTok{ hat, }\DataTypeTok{y =}\NormalTok{ proportion, }\DataTypeTok{fill =} \KeywordTok{factor}\NormalTok{(hat))) }\OperatorTok{+}
\StringTok{  }\KeywordTok{geom_bar}\NormalTok{(}\DataTypeTok{stat =} \StringTok{"identity"}\NormalTok{) }\OperatorTok{+}
\StringTok{  }\KeywordTok{scale_fill_discrete}\NormalTok{(}
    \DataTypeTok{name =} \StringTok{"CI"}\NormalTok{,}
    \DataTypeTok{labels =} \KeywordTok{c}\NormalTok{(}\StringTok{"out"}\NormalTok{, }\StringTok{"in"}\NormalTok{)}
\NormalTok{  ) }\OperatorTok{+}
\StringTok{  }\KeywordTok{xlab}\NormalTok{(}\KeywordTok{expression}\NormalTok{(y))}
\end{Highlighting}
\end{Shaded}

\begin{figure}[H]

{\centering \includegraphics[width=0.7\linewidth]{statistical-computing_files/figure-latex/ciin-1} 

}

\caption{Proportion of $\sigma^2$ in confidence intervals}\label{fig:ciin}
\end{figure}

This leads to empirical confidence level, i.e.~\emph{sample proportion}. Just follow the last step \(\ref{alg:cilevlast}\) of Algorithm \(\ref{alg:algcilev}\).

\begin{Shaded}
\begin{Highlighting}[]
\NormalTok{(ci_lev <-}
\StringTok{  }\NormalTok{ci_lev[,}
\NormalTok{         .(}\DataTypeTok{level =} \KeywordTok{mean}\NormalTok{(hat))])}
\CommentTok{#>    level}
\CommentTok{#> 1: 0.952}
\end{Highlighting}
\end{Shaded}

It is very close to \(0.95\). One of advantages of simulation study is we can assume various situation. For example, \emph{violation of Gausiannity}.

\BeginKnitrBlock{example}[Violation of Normal distribution assumption]
\protect\hypertarget{exm:nonnormal}{}{\label{exm:nonnormal} \iffalse (Violation of Normal distribution assumption) \fi{} }Refer to Example \ref{exm:civar}. This has assumed that \(X_i \stackrel{iid}{\sim}N(\mu = 2, \sigma^2 = 4)\). What if not? For instance,

\[X_1, \ldots, X_n \stackrel{iid}{\sim}\chi^2(df = 2)\]
\EndKnitrBlock{example}

Just change random numbers.

\begin{Shaded}
\begin{Highlighting}[]
\NormalTok{ci_lev2 <-}
\StringTok{  }\KeywordTok{mc_data}\NormalTok{(rchisq, }\DataTypeTok{N =} \DecValTok{20}\NormalTok{, }\DataTypeTok{M =} \DecValTok{1000}\NormalTok{, }\DataTypeTok{df =} \DecValTok{2}\NormalTok{)[,}
\NormalTok{                                            .(}\DataTypeTok{hat =} \KeywordTok{mean}\NormalTok{(}\KeywordTok{ci_var}\NormalTok{(x, }\DataTypeTok{variance =} \DecValTok{4}\NormalTok{, }\DataTypeTok{alpha =} \FloatTok{.05}\NormalTok{))),}
\NormalTok{                                            by =}\StringTok{ }\NormalTok{sam][,}
\NormalTok{                                                      .(}\DataTypeTok{non_normal =} \KeywordTok{mean}\NormalTok{(hat))]}
\end{Highlighting}
\end{Shaded}

\begin{longtable}{r|r}
\caption{\label{tab:emlevpop}Empirical confidence level for each population}\\
\hline
Normal & Chisq\\
\hline
0.952 & 0.763\\
\hline
\end{longtable}

From Table \ref{tab:emlevpop}, we found that \emph{non-normality lowers confidence level} from 0.952 to 0.763.

\hypertarget{hypothesis-tests}{%
\section{Hypothesis tests}\label{hypothesis-tests}}

Using MC method, we have done point estimation and interval estimation. Now consider \emph{hypothesis testing}.

\[H_0: \theta \in \Theta_0 \qquad \text{vs} \qquad H_1: \theta \in \Theta_1\]

where \(\{ \Theta_0, \Theta_1 \}\) is a partition of the parameter space \(\Theta\). First of all, we have \emph{test statistic}

\[T(\mathbf{X}) \stackrel{H_0}{\sim}f\]

and \(f\) is called \emph{null distribution}. Given observed data, we compute this test statistic \(T_0\). Where \(T_0\) is located in the null distribution \(f\) decides whether we reject or accept \(H_0\). If \(T_0\) is very far from the middle, we can say that the realized data set is very rare event under \(H_0\). In this case, we reject \(H_0\). Otherwise, accept it. This is why we compute the tail probability, p-value.

\hypertarget{empirical-p-value}{%
\subsection{Empirical p-value}\label{empirical-p-value}}

\BeginKnitrBlock{example}
\protect\hypertarget{exm:exptest}{}{\label{exm:exptest} }Suppose that \(X_1, \ldots, X_{10} \stackrel{iid}{\sim}Exp(\lambda = 1)\), which are observed as follows

\[`r xexp`\]

Let \(\theta = E(X) = \frac{1}{\lambda}\).

\[H_0: \theta = 0.5 \qquad \text{vs} \qquad H_1: \theta > 0.5\]

Test using \(T = \frac{\overline{X} - \theta_0}{S / \sqrt{n}}\) statistic.
\EndKnitrBlock{example}

Before looking at p-value, briefly look at \emph{empirical null distribution} of test statistic.

\begin{Shaded}
\begin{Highlighting}[]
\KeywordTok{mc_data}\NormalTok{(rexp, }\DataTypeTok{rate =} \DecValTok{2}\NormalTok{)[,}
\NormalTok{                        .(}\DataTypeTok{tstat =} \KeywordTok{t.test}\NormalTok{(x, }\DataTypeTok{mu =} \FloatTok{.5}\NormalTok{)}\OperatorTok{$}\NormalTok{statistic),}
\NormalTok{                        by =}\StringTok{ }\NormalTok{sam] }\OperatorTok\StringTok{ }
\StringTok{  }\KeywordTok{ggplot}\NormalTok{(}\KeywordTok{aes}\NormalTok{(}\DataTypeTok{x =}\NormalTok{ tstat)) }\OperatorTok{+}
\StringTok{  }\KeywordTok{geom_histogram}\NormalTok{(}\DataTypeTok{bins =} \DecValTok{30}\NormalTok{, }\DataTypeTok{col =} \KeywordTok{gg_hcl}\NormalTok{(}\DecValTok{1}\NormalTok{), }\DataTypeTok{alpha =} \FloatTok{.7}\NormalTok{) }\OperatorTok{+}
\StringTok{  }\KeywordTok{geom_vline}\NormalTok{(}\DataTypeTok{xintercept =} \KeywordTok{t.test}\NormalTok{(xexp, }\DataTypeTok{mu =} \FloatTok{.5}\NormalTok{)}\OperatorTok{$}\NormalTok{statistic, }\DataTypeTok{col =} \KeywordTok{I}\NormalTok{(}\StringTok{"red"}\NormalTok{)) }\OperatorTok{+}\StringTok{ }\CommentTok{# xexp: observed data}
\StringTok{  }\KeywordTok{geom_vline}\NormalTok{(}\DataTypeTok{xintercept =} \OperatorTok{-}\KeywordTok{t.test}\NormalTok{(xexp, }\DataTypeTok{mu =} \FloatTok{.5}\NormalTok{)}\OperatorTok{$}\NormalTok{statistic, }\DataTypeTok{col =} \KeywordTok{I}\NormalTok{(}\StringTok{"red"}\NormalTok{)) }\OperatorTok{+}
\StringTok{  }\KeywordTok{xlab}\NormalTok{(}\StringTok{"T"}\NormalTok{)}
\end{Highlighting}
\end{Shaded}

\begin{figure}[H]

{\centering \includegraphics[width=0.7\linewidth]{statistical-computing_files/figure-latex/expemp-1} 

}

\caption{Emprirical Null Distribution}\label{fig:expemp}
\end{figure}

By proceeding the similar way, we can get empirical distribution of test statistics. Some are out of observed \(T_0\), some are not. Motivation is that we just count these. Proportion of these would estimate p-value. Recap what p-value is.

\BeginKnitrBlock{definition}[p-value]
\protect\hypertarget{def:pval}{}{\label{def:pval} \iffalse (p-value) \fi{} }Let \(T\) be test statistic and let \(T_0\) be observed test statistic given data. Then p-value is

\[
p-value := \begin{cases}
  P(\lvert T \rvert \ge T_0 \mid H_0) & \text{both sided} \\
  P(T \ge T_0 \mid H_0) & \text{one sided} \\
  P(T \le T_0 \mid H_0) & \text{one sided}
\end{cases}
\]
\EndKnitrBlock{definition}

Denote that p-value is probability. So in MC setting, we can estimate this by computing \emph{sample mean of identity function}.

\BeginKnitrBlock{lemma}[Empirical p-value]
\protect\hypertarget{lem:emppval}{}{\label{lem:emppval} \iffalse (Empirical p-value) \fi{} }Let \(T_0\) be observed test statistic and let \(\{ T_1, \ldots, T_M \}\) be test statistic computed in each MC sample.

\[
\text{Empirical p-value} = \begin{cases}
  \frac{\Big\lvert \{ T_j : (T_j > \lvert T_0 \rvert) \:\text{or}\: (T_j < -\lvert T_0 \rvert) \} \Big\rvert}{M} & \text{both-sided} \\
  \frac{\Big\lvert \{ T_j : (T_j > T_0 ) \} \Big\rvert}{M} \:\text{or}\: \frac{\Big\lvert \{ T_j : (T_j < T_0 ) \} \Big\rvert}{M} & \text{one-sided}
\end{cases}
\]
\EndKnitrBlock{lemma}

\begin{algorithm}[H] \label{alg:algpval}
  \SetAlgoLined
  \SetKwInOut{Input}{input}
  \SetKwInOut{Output}{output}
  \Input{Given observed data, compute $T_0$}
  \For{$m \leftarrow 1$ \KwTo $M$}{
    Generate $X_1^{(m)}, \ldots, X_n^{(m)} \stackrel{H_0}{\sim}f$\;
    Compute $T_m(\mathbf{X}^{(m)})$\;
  }
  Empirical p-value $\hat{p} = \begin{cases} \frac{\Big\lvert \{ T_j : (T_j > \lvert T_0 \rvert) \:\text{or}\: (T_j < -\lvert T_0 \rvert) \} \Big\rvert}{M} & \text{both-sided} \\ \frac{\Big\lvert \{ T_j : (T_j > T_0 ) \} \Big\rvert}{M} \:\text{or}\: \frac{\Big\lvert \{ T_j : (T_j < T_0 ) \} \Big\rvert}{M} & \text{one-sided} \end{cases}$\; \label{alg:empp}
  \Output{$\hat{p}$}
  \caption{Empirical p-value by Monte Carlo method}
\end{algorithm}

Go back to Example \ref{exm:exptest}. Only left is computing \(\ref{alg:empp}\) of Algorighm \(\ref{alg:algpval}\). (Denote that \texttt{xexp} in the code is vector object of observed data).

\begin{Shaded}
\begin{Highlighting}[]
\NormalTok{(tt_exp <-}
\StringTok{  }\KeywordTok{mc_data}\NormalTok{(rexp, }\DataTypeTok{rate =} \DecValTok{2}\NormalTok{)[,}
\NormalTok{                          .(}\DataTypeTok{tstat =} \KeywordTok{t.test}\NormalTok{(x, }\DataTypeTok{mu =} \FloatTok{.5}\NormalTok{)}\OperatorTok{$}\NormalTok{statistic),}
                          \DataTypeTok{by =}\NormalTok{ sam][,}
\NormalTok{                                    .(}\DataTypeTok{pval =} \KeywordTok{mean}\NormalTok{(tstat }\OperatorTok{>}\StringTok{ }\KeywordTok{abs}\NormalTok{(}\KeywordTok{t.test}\NormalTok{(xexp, }\DataTypeTok{mu =} \FloatTok{.5}\NormalTok{)}\OperatorTok{$}\NormalTok{statistic)))])}
\CommentTok{#>     pval}
\CommentTok{#> 1: 0.046}
\end{Highlighting}
\end{Shaded}

It is smaller that \texttt{0.05}, so we reject \(H_0\).

\hypertarget{comparing-several-tests}{%
\subsection{Comparing several tests}\label{comparing-several-tests}}

MC method would be used in comparing tests rather than conducting test itself. By generating random number, we can evaluate tests.

\[H_0: \theta \in \Theta_0 \qquad \text{vs} \qquad H_1: \theta \in \Theta_1\]

As mentioned earlier, \(\{ \Theta_0, \Theta_1 \}\) is a partition of the parameter space \(\Theta\). For this test, we can perform several tests. Test method 1, test method 2, et cetera. All these methods produce error, but these errors might be different. So we try to compare this.

\begin{longtable}[]{@{}ccc@{}}
\toprule
what is true & accept \(H_0\) & reject \(H_0\)\tabularnewline
\midrule
\endhead
\(H_0\) & correct decision & \emph{Type I error}\tabularnewline
\(H_1\) & \emph{Type II Error} & correct decision\tabularnewline
\bottomrule
\end{longtable}

In most tests, we aims to reject \(H_0\). By rejecting it, we can evidently say that \(H_0\) is not true. In this sense, we treat type I error more importantly that type II error in general. Test strategy becomes to control type I error probability first and then lower type II error probabilty.

\BeginKnitrBlock{definition}[Power function]
\protect\hypertarget{def:bpower}{}{\label{def:bpower} \iffalse (Power function) \fi{} }Let \(\theta \in \Theta\) be a parameter of a test.

\[\beta(\theta) := P(\text{reject}\: H_0 \mid \theta)\]
\EndKnitrBlock{definition}

With this power function, each type I error and type II error probability is given.

\BeginKnitrBlock{lemma}[typeerr]
\protect\hypertarget{lem:typeerr}{}{\label{lem:typeerr} \iffalse (typeerr) \fi{} }

\begin{enumerate}
  \item $P(\text{Type I error}) = \beta(\theta_0), \quad \theta_0 \in \Theta_0$
  \item Power $\beta(\theta_1) = 1 - P(\text{Type II error}), \quad \theta_1 \in \Theta_1$
\end{enumerate}
\EndKnitrBlock{lemma}

Following our test strategy, fixing \(P(\text{Type I error})\) and maximizing \(\beta(\theta_1)\), we construct following test.

\BeginKnitrBlock{definition}[Size $\alpha$ Test]
\protect\hypertarget{def:sizetest}{}{\label{def:sizetest} \iffalse (Size \(\alpha\) Test) \fi{} }A test with \(\beta(\theta)\) is called size \(\alpha\) test if and only if

\[\alpha := \sup_{\theta \in \Theta_0} \beta(\theta), \quad 0 \le \alpha \le 1\]
\EndKnitrBlock{definition}

\begin{figure}[H]

{\centering \includegraphics[width=0.7\linewidth]{statistical-computing_files/figure-latex/sizealpha-1} 

}

\caption{Size $\alpha$ Test}\label{fig:sizealpha}
\end{figure}

Then how to compare tests? Look at the following example. Three columns of the middle part are type I error rate.

\begin{longtable}[]{@{}ccccc@{}}
\toprule
test methods & \(\alpha = 0.01\) & \(\alpha = 0.05\) & \(\alpha = 0.01\) & Power\tabularnewline
\midrule
\endhead
Test 1 & 0.09 & 0.04 & 0.01 & 0.7\tabularnewline
Test 2 & 0.11 & 0.06 & 0.01 & 0.65\tabularnewline
Test 3 & 0.15 & 0.07 & 0.02 & 0.9\tabularnewline
\bottomrule
\end{longtable}

Here, we will choose \textbf{Test 1}.

\begin{enumerate}
\def\labelenumi{\arabic{enumi}.}
\tightlist
\item
  \(\text{Type I error rate} \approx \alpha\)

  \begin{itemize}
  \tightlist
  \item
    before looking at power, this should be satisfied.
  \item
    So Test 3 is excluded
  \end{itemize}
\item
  Larger power

  \begin{itemize}
  \tightlist
  \item
    Thus, we select Test 1.
  \end{itemize}
\end{enumerate}

\hypertarget{empirical-type-i-error-rate}{%
\subsection{Empirical type-I error rate}\label{empirical-type-i-error-rate}}

Recall Lemma \ref{lem:typeerr}. As in p-value, we just compute sample proportion for each type I error rate and power under null and alternative distribution.

\BeginKnitrBlock{lemma}
\protect\hypertarget{lem:testnull}{}{\label{lem:testnull} }Consider \(H_0: \theta \in \Theta_0 \qquad \text{vs} \qquad H_1: \theta \in \Theta_1\).

Define \(I(\mathbf{X})\) by

\[
I(\mathbf{X}) = \begin{cases}
  1 & H_0 \:\text{is rejected} \mid H_0 \\
  0 & otherwise
\end{cases}
\]

For each MC sample, compute this statistic \(I_m = I(\mathbf{X}^{m})\). Then empirical type I error rate can be computed as

\[\frac{1}{M}\sum_{m = 1}^M I_m\]
\EndKnitrBlock{lemma}

\begin{algorithm}[H] \label{alg:algtype1}
  \SetAlgoLined
  \SetKwInOut{Input}{input}
  \SetKwInOut{Output}{output}
  \Input{$H_0: \theta \in \Theta_0 \quad \text{vs} \quad H_1: \theta \in \Theta_1$}
  \For{$m \leftarrow 1$ \KwTo $M$}{
    Generate $X_1^{(m)}, \ldots, X_n^{(m)} \stackrel{H_0}{\sim}f$\;
    Compute $T_m(\mathbf{X}^{(m)})$\;
    Compute $I_m = \begin{cases} 1 & H_0 \:\text{is rejected} \mid H_0 \\ 0 & \text{otherwise} \end{cases}$\;
  }
  Empirical Type I error rate $\hat\alpha = \frac{1}{M}\sum\limits_{m = 1}^M I_j$\;
  \Output{compare $\hat\alpha$ with $\alpha$}
  \caption{Empirical type I error rate by Monte Carlo method}
\end{algorithm}

\hypertarget{empirical-power}{%
\subsection{Empirical power}\label{empirical-power}}

\hypertarget{count-five-test-for-equal-variance}{%
\subsection{Count Five test for equal variance}\label{count-five-test-for-equal-variance}}

\hypertarget{statistical-methods}{%
\section{Statistical Methods}\label{statistical-methods}}

\hypertarget{bootstrap}{%
\section{Bootstrap}\label{bootstrap}}

\bibliography{book.bib,packages.bib}


\end{document}
